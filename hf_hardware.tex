\section{Experimental Setup}

\subsection{Source Driver System}
To perform calibration of the calorimeter with a radioactive source, a
specialized source driver system has been developed. It includes devices capable of
inserting a long thin wire, approximately 11\unit{m} long, tipped with a point-like
radioactive source into the HF source tubes embedded in the calorimeter. As the
radioactive source moves through the HF absorber, gamma rays are generated and
consequently create Compton electrons, which in turn can generate
Cherenkov photons inside the quartz fibers if they exceed the Cherenkov threshold.
With the new HF PMTs, the quantum efficiency is such that only 30\% of the photons
reaching the PMT cathode face will actually be converted into the read out signal.

The thin wire, referred to as the source wire, used by the system is made of
stainless steel and has an inner and outer diameter of 0.406 mm and 0.711 mm,
respectively. The end of the wire that penetrates the HF absorber, the front end,
is melted shut, shaped into a bullet nose, and chemically plated to reduce friction.
The point-like radioactive source is inserted into the opposite end of
the wire, the back end, and held in place against the front end by a fine steel
piano wire. The outer source wire and inner piano wire are then crimped together at
the back end in order to fix the position of the radioactive source.

A Lexan polycarbonate reel, belt driven by a DC reversible electric motor, is used
to insert or retract the source wire into or out of the calorimeter's source tubes.
The calorimeter's source tubes are coupled to acetal plastic tubing to mediate the
transfer of the source wire from the source driver into the absorber. The
transition between the plastic tube and the metal source tube typically involve
small-angle conical holes in brass that channel the source wire to the source tube.

The driver system also contains an additional electric motor that functions to
select the source tube into which to direct the source wire, an action referred
to as indexing. The position of the radioactive source, relative to the source
driver and referred to as the reel position, is provided by an optical rotary
encoder read out by industrial batch counters. Typical speeds at which the
radioactive source may be inserted or retracted by the driver are between 5 and
15\unit{cm/s}. Figure~\ref{fig:Source_Driver} displays the source driver configuration.

\begin{figure}[htb]
   \begin{center}
      \includegraphics[width=.5\textwidth]{figures/ch_hfcalibration/Source_Driver.png}
      \caption{The source wire is spooled on a reel sitting horizontally near
      the brass-clad pig that houses the radioactive end of the wire during storage.
      The vertically-oriented black square plate is the indexer system. The DC electric
      motor that functions with the indexer is seen in top left. Two nylon source tubes
      are connected to the indexer on the left.}
      \label{fig:Source_Driver}
   \end{center}
\end{figure}

\subsection{HF DAQ Description}
The PMT analog signals are read out by QIEs, standing for charge (Q),
integration (I), and encode (E). Each such QIE has 3 differential inputs, which
allows to digitize 3 PMTs simultaneously. Differential inputs are used to subtract
any externally induced noise in signal cables. Each QIE has also 1 fiber-optic
output, which transfers the digitized information to HTRs (HCAL Trigger and
Readout). The output of 8 QIEs (8 fibers) is bundled up into 1 cabel, which in
turn is connnected to either top or bottom half of the HTR, equaling 16 QIEs
(16 fibers) per HTR. Therefore, each HTR receives input from 48 PMTs: 24 PMTs for
each tower's HAD channel and 24 PMTs for each tower's EM channel.

A large dynamic range in PMT signal processing is acquired using
a multi-range technique. The input current is sumulatenously integrated on all
four ranges, and comparators are used to isolate the lowest range that is not at
full scale. The selected voltage representing the integrated charge is then put
through an on-chip piecewise linear Flash ADC (Analogue to Digital Converter),
with bins weighted according to the time slice (TS) firmware used. For 1TS, 15
bins are weighted 1, 7 bins weighted 2, 4 bins weighted 3, 3 bins weighted 4,
and 3 bins weighted 5, providing a range from 0 to 63 ADC counts, with the last
bin (overflow bin) containing all charge above 63 ADC counts. For 2TS, 10 bins
are weighted 1, 6 bins weighted 2, 5 bins weighted 4, 5 bins weighted 8, 3 bins
weighted 16, 2 bins weighted 32, and 1 bin weighted 62, providing a range from
0 to 194 ADC counts, with the overflow bin containing all charge above 194 ADC
counts. Operations are time multiplexed and pipelined to allow signals to
settle and to make the reset interval the same as the inegration interval.
Clocking is provided at the frequency of 40\unit{MHz}, with a latency of
100\unit{ns}, as the pipeline is four clock cycles deep. Each QIE contains a
set of four capacitors, and only one capacitor is acquiring charge during a
given clock cycle. The output is a 5-bit mantissa representing the voltage on
the particular capacitor, and a 2-bit exponent indicating the range
represented by a coded address of the capacitor.

The firmware implemented during radioactive source data collecting uses a
given TS histogramming mode and operating voltage (OV). In the histogramming
mode, the absorber's response charge is represented across 32 histogram bins,
with the final bin, Bin 32, being the overflow bin containing all data with
charge beyond the particular range. For 1TS firmware, each TS was recorded in
the histogram. For 2TS firmware, two adjacent TS segments were compared, and
the maximum output from the pair was recorded in the histogram. To accomodate
the 2TS maximum outputs, the expanded QIE bin range described above was used
with respect to 1TS.

Each capacitor from a QIE set fills a separate histogram at the sampling rate,
i.e. latent cycle, and once $6.5535 x 10^4$ samples are collected the histogram
is read out by the DAQ, two HTR fibers from each HTR half at a time. Data is
saved by event, each event containing the histograms read out by all powered HTR
halves - a given PMT's output is stored in every fourth event. This provides
that each event represents 6.55\unit{ms} of source data. For 1TS, each sample
was collected over 25\unit{ns}, and for 2TS each sample was collected over
50\unit{ns}.