\section{Results and Discussion}
\subsection{2013 Results}
As it has already been pointed out, outlier channels from Figure~\ref{fig:QIE_Slope} have been excluded.
Moreover, to achieve a result for the average energy deposition per channel type,
EM or H, from the radioactive source, we only considered data from towers with
IEta $<$ 35 to minimize the effects of radiation damage of HF fibers at higher $\eta$ towers, shown in Figure~\ref{fig:PMT_Drift}.
\begin{figure}[htb]
   \begin{center}
      \includegraphics[width=.9\textwidth]{figures/ch_hfcalibration/PMT_Drift.png}
      \caption{R7525 PMT relative signal strength with respect to 10 Feb. 2011 as a
      function of time since that date, for various $\eta$ locations. The solid
      black curve represents the integrated luminosity within CMS over the same
      time period}
      \label{fig:PMT_Drift}
   \end{center}
\end{figure}

Figure~\ref{fig:HFM_2013_Res} shows the results from 2013 when only the near side
of HF- was sourced, nine wedges, and only the towers below IEta = 35 are
considered, eight towers per wedge. The average energy deposition extracted from
2013 sourcing campaign, for EM and H channels separately, is 744.6 $\pm$ 6.3 keV and 706.8 $\pm$ 7.7 keV per time slice, respectively. There is a 5\% difference observed between the EM and HAD channels, while their
respective precision is kept within 1 \% each.
\begin{figure}[htb]
   \begin{center}
      \includegraphics[width=.45\textwidth]{figures/ch_hfcalibration/HFM_2013_Res_EM.png}
      \includegraphics[width=.45\textwidth]{figures/ch_hfcalibration/HFM_2013_Res_H.png}
      \caption{(a) EM Energy deposition for each tower below IEta = 35. (b) H Energy deposition for the same towers}
      \label{fig:HFM_2013_Res}
   \end{center}
\end{figure}

\subsection{2014 Results}
The source signals, ${\langle{Q}\rangle}^{Geom}_{c}$, for HF+ and HF- for 2014 data with new PMTs, corrected for geometry (geometry containment factor), firmware used (1 TS or 2 TS), Operating Voltage to be used (converting to OV2 from either OV1 or OV1+100), are calculated using Equation~\ref{eq:Sig_OV2} and are presented in Figure~\ref{fig:Signal_@OV2_TT0_withoutOF_FORDN}.
\begin{figure}[htb]
	\begin{center}
		\includegraphics[width=.6\textwidth]{figures/ch_hfcalibration/Signal_@OV2_TT0_withoutOF_FORDN.png}
		\caption{Actual Signal from the Source recorded by the PMT at OV2 (Operational Voltage 2)}
		\label{fig:Signal_@OV2_TT0_withoutOF_FORDN}
	\end{center}
\end{figure}

The HF Gains, ${CC}^{Run II}_{c}$ in units of GeV/ADC, for HF+ and HF- are
computed using Equation~\ref{eq:HF_Gains} and are presented in Figure~\ref{fig:ADC2GeV_OV2_TT0_withoutOF_FORDN}. In Equation~\ref{eq:HF_Gains}, if $c$ is a EM channel from a given tower, then use the EM result from 2013, and similar logic applies for the H channel counterparts.
\begin{figure}[!h]
	\begin{center}
		\includegraphics[width=.6\textwidth]{figures/ch_hfcalibration/ADC2GeV_OV2_TT0_withoutOF_FORDN.png}
		\caption{Distribution of Calibration Coefficients (HF Gains)}
		\label{fig:ADC2GeV_OV2_TT0_withoutOF_FORDN}
	\end{center}
\end{figure}

To account for the differences in Operating Voltages for Sourcing vs Run II Physics campaign, the PMT Gain Ratios were applied as Conversion Factors, distributions of which can be found in Figure~\ref{fig:PMT_Gains}
\begin{figure}[!h]
	\begin{center}
		\includegraphics[width=.5\textwidth]{figures/ch_hfcalibration/GainRatios.png}
		\caption
		{Black - Ratio of PMT Gains for OV2/OV1. Red - Ratio of PMT Gains for OV2/OV1+100}
		\label{fig:PMT_Gains}
	\end{center}
\end{figure}

% \subsection{HF Gains and CondDB Submission}
% To finalize the HF Source Calibration, we had to upload the computed calibration
% coefficients, ${CC}^{Run II}_{c}$, into the Conditions DataBase (CondDB).
% However, up until now, the units we used to present them were GeV/ADC, whereas
% for submission it is required to provide calibration coefficients in GeV/fC.

% \begin{figure}[htb]
% 	\begin{center}
% 		\includegraphics[width=.5\textwidth]{figures/ch_hfcalibration/QIE_Slopes_Range0.png}
% 		\caption{Distribution of Range 0 QIE Slopes}
% 		\label{fig:QIESlopes}
% 	\end{center}
% \end{figure}

% Therefore, we first extracted the ADC to fC conversion factors from the CondDB,
% which are also called "QIE Slopes". As we have only been using Range 0 for
% calibration purposes, they are the only ones of interest to us. At the time when
% the actual analysis was performed, the half of the QIE Slopes were still mixed
% up, as a consequence we averaged out all the slopes and used the mean for
% converting our GeV/ADC to GeV/fC. In the Figure~\ref{fig:QIESlopes}, the
% distribution of Range 0 QIE Slopes is presented. The obtained mean of 0.3593
% (ADC/fC), as it was already pointed out, has been used to calculate the "final"
% calibration coefficients in units of GeV/fC, which are presented in the
% Figure().
% Using the mean of Range 0 QIE Slopes adds an additional
% uncertainty of 2\%, which can be deduced from the spread in the distribution of
% QIE Slopes in the Figure~\ref{fig:QIESlopes}.

