
% Appendices
\appendix \label{sec:Appendix}

\chapterx[Formatting Appendices]{Formatting Appendices}

To avoid having multiple appendix files, we can insert a new page.
In \LyX{}, Insert $\rightarrow$ Formatting $\rightarrow$ New Page.
In \LaTeX{}, simply add \texttt{\textbackslash{}newpage\{\}}. Immediately
following the new page, add \texttt{\textbackslash{}chapter\{...\}}
and continue as if it were a new file.

\newpage{}

\chapter{Table of Contents Issues}


\section{Extra Vertical Spaces}

The thesis class file contains code to format the TOC in the manner
described by the thesis manual. However, there is one caveat. Chapters
without subheadings (e.g., Chapter 5) produce an extra space. As this
rarely happens, it probably went overlooked by the previous authors.
To remedy the situation, a new chapter type is available; \texttt{\textbackslash{}chapterx{[}$title${]}\{$title$\}}.
Both options are required in order for this to work correctly (i.e.,
both {[}{]} and \{\} must contain the title).


\section{Titles Running into the Page Number Column}

If not an ntheorem entry, use the alt header text and add the appropriate
\TeX{} commands:

\noindent \texttt{\textbackslash{}chapter{[}Cyclic-Order Neighborhoods
for Vehicle Routing\textbackslash{}\textbackslash{}Problems{]}\{Cyclic-Order
Neighborhoods for Vehicle Routing Problems\} \textbackslash{}label\{chp:CNS\}}

If an ntheorem command, the formatting will not work for some strange
reason. What does is adding a bunch of math spaces \texttt{\$\textasciitilde{}\textasciitilde{}\textasciitilde{}\textasciitilde{}\textasciitilde{}\textasciitilde{}\textasciitilde{}\textasciitilde{}\textasciitilde{}\$}
at the break point until the rest of the line is pushed to the next
line.

\newpage{}

\chapter{Example of the Heading Scheme}\label{appendix:headings}

The thesis manual (page 31) requires that all headings must be differentiated
from one another via formatting and they can be no longer than 4 inches
in length. The original uithesis.cls file only reached third-order
headings and did not even cover the length issue. I have extended
this to fifth-order headings to bring the number of headings to six.
The formatting is as follows.
\begin{enumerate}
\item Chapter: centered and all caps
\item Section: centered and bold
\item Subsection: centered
\item Subsubsection: left-aligned and bold
\item Paragraph: left-aligned
\item Subparagraph: left-aligned and italic
\end{enumerate}
I have only been able to come up with a work-around for the 4 inch
requirement and not a true fix. In your code, wrap the section(s)
that is(are) too long with a minipage of size 4 inches. You will need
to adjust the alignment and the vspace before and after to match the
current subheading. The spacing below seems to work in most cases
for the section subheading. As this can be a little bit annoying to
perfect, only use it for headings you know are over four inches.

~

\begin{singlespace} % for output formatting only
\begin{verbatim}
\begin{center}
\begin{minipage}[c]{4in}
\begin{singlespace}
\vspace{5.2mm}
\section...
\end{singlespace}
\end{minipage}
\end{center}
\vspace{1mm}
\end{verbatim}
\end{singlespace}

\noindent Below is an example of a subheading that's too long and
the fix.


\section{This Title is Longer than the Alloted Four Inch Maximum}

\begin{center}\begin{minipage}[c]{4in}\begin{singlespace}\vspace{5.2mm}
\section{This Title is Longer than the Alloted Four Inch Maximum}
\end{singlespace}\end{minipage}\end{center}\vspace{1.1mm}

\noindent Below is an example of all headings below chapter which
can be seen at the top of this page.


\section{First-Order (section)}


\subsection{Second-Order (subsection)}


\subsubsection{Third-Order (subsubsection)}


\paragraph{Fourth-Order (paragraph)}


\subparagraph{Fifth-Order (subparagraph)}

\newpage{}

\chapter{Important Points from the Thesis Manual}


\section{Mechanics}

\noindent Text Size
\begin{itemize}
\item 12-pt font for all preliminary pages, text, table and figure captions,
appendices\textquoteright{} headings, references, and page numbers
(default in uithesis.cls)
\item Table and figures proper and their footnotes, appendices, and equations
may be reproduced in different size and style fonts than that of the
thesis proper
\item Super- and subscript numbers and letters and footnotes that appear
at the bottom of the page may be printed in 10-point typeface (default
in uithesis.cls)
\end{itemize}
Style
\begin{itemize}
\item Boldface may be used for the title of the thesis, all major headings,
within the formatting scheme for subheadings, and for all table or
all figure captions
\item The use of SMALL CAPS is not allowed within any part of the thesis
\item Italics may be used as part of the subheading scheme
\item Italics may also be used sparingly for emphasis, foreign words, technical
or key terms, mathematical expressions, or book and journal titles
\end{itemize}
Margin
\begin{itemize}
\item left = 1.5 inch (default in uithesis.cls)
\item right, top, bottom = 1 inch (default in uithesis.cls)
\end{itemize}
~

\noindent Text Spacing
\begin{itemize}
\item Double spacing (default in uithesis.cls)

\begin{itemize}
\item 24-point
\item 2:1 ratio, the line spacing should be set such that no less than 5
and no more than 6 single-spaced lines (text or blank lines) are positioned
within 1 vertical inch (default in uithesis.cls)
\end{itemize}
\item Single spacing

\begin{itemize}
\item 12-point
\item Single spacing is a requirement for footnotes to text, tables and
figures; for bibliographic entries. A double space should be used
to separate individual footnote and reference citations. (default
in uithesis.cls)
\item Multi-line major headings, subdivisions, and table and figure captions
as well as appendix material and information within tables and figures
may be single spaced if done in a consistent manner. (default in uithesis.cls)
\item Direct quotations should be single spaced and indented.
\end{itemize}
\end{itemize}
Pagination
\begin{itemize}
\item All page numbers on the final-deposit copies must appear in same 12
point type font used in the thesis proper. (default in uithesis.cls)
\item Preliminary Pages (default in uithesis.cls)

\begin{itemize}
\item Number consecutively in lowercase roman numerals beginning with \textquotedblleft{}ii
on the first page following the Certificate of Approval.
\item Neither the copyright page nor the Certificate of Approval is counted
or numbered.
\item Numbers are centered without punctuation 0.5 inches from the bottom
of the page.
\end{itemize}
\item Text and Reference pages (default in uithesis.cls)

\begin{itemize}
\item Number consecutively in arabic numerals, beginning with 1 on the first
page following the preliminaries.
\item Numbers are placed without punctuation in the upper right-hand corner
1 inch from the right and 0.5 inches from the top of the page.
\end{itemize}
\end{itemize}

\section{Parts of the Thesis}
\begin{itemize}
\item Abstract title page (external): mandatory for PhD and DMA thesis candidates
only (include phd in document class)
\item Doctoral abstract (external): mandatory for PhD and DMA thesis candidates
only (include phd in document class)

\begin{itemize}
\item cannot contain graphs, charts, tables, or illustrations
\item two-page maximum includes the abstract text and the approval lines
\end{itemize}
\item Title page: mandatory (default)

\begin{itemize}
\item Include only the academic rank \textendash{} but not academic degrees
\textendash{} of the faculty member (e.g., Associate Professor John
Doe).
\end{itemize}
\item Copyright page: optional (include copyrightpage in document class)
\item Certificate of Approval: mandatory (default)
\item Dedication: optional (include dedicationpage in document class)

\begin{itemize}
\item It customarily begins with the word \textquotedblleft{}To,\textquotedblright{}
and has no ending punctuation.
\end{itemize}
\item Epigraph or Frontispiece: optional (include epigraphpage in document
class)
\item Acknowledgments: optional (include ackpage in document class)
\item Abstract (internal): optional (include abstractpage in document class)
\item Table of Contents: mandatory (default)

\begin{itemize}
\item Neither underlining, use of boldface or italics for stylistic purposes
in text, nor reference numbers appearing with text headings are placed
in the Table of Contents listing.
\end{itemize}
\item List of Tables: when appropriate (include tables in document class)
\item List of Figures: when appropriate (include figures in document class)
\item List of Symbols/Abbreviations, etc.: when appropriate (not implemented)
\item Preface: optional (not implemented that I know of)
\end{itemize}

\section{Additional information you should know about selected sections}

\noindent Appendix
\begin{itemize}
\item Appendices normally precede the bibliography
\end{itemize}
Notes at the End of each Major Division
\begin{itemize}
\item The notes always begin on a new page. The title, \textquotedblleft{}Notes,\textquotedblright{}
is placed at the top, 1-inch margin; the established first-order subdivision
style is used to format the subheading.
\item Exception: In a numbered subdivision scheme the subheading,\textquotedblleft{}Note,\textquotedblright{}
is not preceded by a designator (number).
\end{itemize}
Notes at the End of the Thesis
\begin{itemize}
\item All notes are collected and placed within a major division titled
NOTES. This heading is typed in all capital letters, centered 1inch
from the top of the page, with the first entry beginning no less than
a double space below.
\item The section NOTES will always directly precede the bibliography.
\item If notes are arranged by chapters within the NOTES section, it is
customary to separate the notes for each chapter by first-order subdivision
titles identifying the chapters. These first-order subdivisions follow
each other without gaps in text.
\end{itemize}

\section{Tables and Figures}

\noindent Placement
\begin{itemize}
\item According to the practice of the department tables/figures may appear
in the text near the point of reference, may be grouped at the end
of each major division (chapter) in which they are mentioned, or presented
in appendices.
\item Tables/figures appearing on the same page as other material (text,
table/figure, etc.) must begin and end on the same page, and must
be separated from surrounding material above and below by at least
three blank, single-spaced lines. (default in uithesis.cls)
\item Tables/figures may be continued to an additional page(s) only if the
initial table/figure page is devoid of other material.
\item Several short tables/figures may be grouped on a single page, provided
that all are started and completed on the same page and that at least
three blank, single-spaced lines separate one from the other.
\end{itemize}
Numbering
\begin{itemize}
\item Each table/figure is assigned a unique number in the order of physical
appearance in the thesis. (default in uithesis.cls)
\item All tables are numbered in one series within the text; figures are
numbered in a separate sequential series. (default in uithesis.cls)
\item The author may elect to use a consecutive arabic or roman numeral
series or a doublenumbering system (1.1, 1.2, 2.1) when creating table/figure
designators. (default in uithesis.cls)
\item For all numbering schemes, tables/figures appearing in appendices
are to be numbered separately from text table/figures (e.g., A1, A2,
B1, B2, etc.). (default in uithesis.cls)
\end{itemize}
Captions
\begin{itemize}
\item Every table/figure must bear a caption that consists of a number preceded
by the word \textquotedblleft{}Table\textquotedblright{} (\textquotedblleft{}Figure\textquotedblright{})
and followed by a descriptive title. (default in uithesis.cls)
\item Captions must be typed in the same size and style font used throughout
the thesis text, although a figure/table proper may be reproduced
in another type font, reduced, or enlarged to fit on the page.
\item Captions throughout must be all single- or double-spaced.
\item Capitalization, punctuation, and layout of the captions must be consistent
for all tables/ figures in the series, though the style of table captions
may differ from the style of figure captions.
\item If a table/figure continues to one or more following pages, then the
figure/table number and a \textquotedblleft{}continued\textquotedblright{}
notation (e.g., Table 3\textemdash{}continued) appears on each page
after the first. The descriptive title is not repeated in part or
full on continuation pages. (see long tables)
\end{itemize}
Tables (additional specifications)
\begin{itemize}
\item All tables must be either boxed or exhibit full-width horizontal beginning
and end lines.
\item Footnotes are placed below the end line. Tables should have self-contained
footnotes that do not rely on the referencing system used for the
thesis text. A new series of footnote numbers or symbols is begun
for each table; the chosen style should be used consistently for all
tables throughout the thesis. Individual footnotes are to be single-spaced,
with double spacing between notes.
\end{itemize}
Figures (additional specifications)
\begin{itemize}
\item Keep in mind that although use of color (including color photographs)
is acceptable, colors become shades of gray in copying and microfilming.
\item Unlike tables, supplementary descriptive information may be included
as part of figure captions.
\end{itemize}