\section{Introduction} \label{section:simulations_introduction}

\subsection{Simulation Objective} \label{subsection:simulations_introduction_objective}
Any simulation is an approximation of a certain process. It is a model that one builds to try to understand something under investigation. It can come in various kinds with different names: virtual reality, avia simulator, computer game or be it any physics simulation. In the end, they are all trying to help model a process of interest. Therefore, the primary objective of any simulation is to model a certain process as good as this process actually is in reality.

Physics Simulations are not the exceptions - it is how Experimental High Energy Physics Experiments function. One starts out by having an objective to measure some quantity or to verify a hypothesis. Typically that means that a model already exists that provides with signatures and features to look for. Then, a simulation of a system is built used to measure physical quantities of interest and then to compare experimental results with simulated predictions. This final step is the most important - if the simulation (model) has not been possible, one would not have something to compare results with and therefore no way to draw conclusions.

\subsection{Geant4} \label{subsection:simulations_introduction_geant4}
For the purpose of simulating High Energy Physics (HEP) Calorimeter systems, {\sc Geant4} \cite{Geant4-1,Geant4-2} is used - a software toolkit which provides an interface for building actual simulation, carrying it out and collecting the results. To be more precise, every simulation has the following features:
\begin{itemize}
    \item \textbf{Materials}. All the materials that are to be used in the simulation must be defined. All the properties of all of the chemical elements must be also defined. Important to point out, that if for instance one has a material that acts as a scintillator in a calorimeter, optical characteristics of such a material must be provided separately and can be optimized.
    \item \textbf{Geometry}. Using various geometrical primitives (cube, cone, etc...) a geometrical specification of the simulation has to be provided. Technically speaking, \textbf{Materials} $+$ \textbf{Geometry} define the physical layout of a calorimeter.
    \item \textbf{Physics}. One of the most important parts of a simulation is the ability to change the physics processes that are being used and see how that affects calorimeter's response. For instance, if there is a scintillation material that is expected to output light upon incoming radiation, then by turning off Scintillation Physics Process, the output of the calorimeter will be suppressed. In {\sc Geant4}, this is typically achieved via \textbf{Physics Lists} - that contain the specification of the most important physics processes.
    \item \textbf{Readout}. The whole idea of performing a simulation is to yield some output that you can then analyze further. For that, there are \textbf{Sensitive Detectors}, which get attached to \textbf{Geometry Volumes} and get triggered for every single step of any particle within that volume. What gets stored is up to the user to specify.
    \item \textbf{Primary Generators}. A special case of the physics is the generation of a primary vertex. For instance, typically, when the performance of a calorimeter is studied, a precise control of the setup is required and it is achieved by shooting a single final state particle into the system. This procedure allows to deduce basic performance characteristics of the system under study.
    \item \textbf{Simulation Engine}. Finally, {\sc Geant4} provides an engine to carry out the simulation itselt: track all the particles for each step, trigger various transitions for readout, apply the physics processes, etc...
\end{itemize}