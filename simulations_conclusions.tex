\section{Conclusions} \label{section:simulations_conclusions}
Simulations of severel standalone systems have been built: High Granularity Calorimeter and Shashlik/HE System. Baseline performance (Linearity and Energy Resolution) has been established and cross-checked with other models. %Finally we scaled both systems up to the production CMS dimensions and evaluated the reconstruction efficiency.

\section{Future Prospects}
Standalone Simulations of Calorimeters is an ideal playground for optimizing the parameters of future systems. It avoids unneccessary overhead, which is always introduced once you try to scale things up and use very precise simulations like CMS. The time to physics gets substantially reduced and allows to accelerate the development of reconstruction techniques. A very interesting future use case is Machine Learning (ML) Benchmarking. In comparison, Mnist Digit Classification has become the defacto benchmarking for the field of Computer Vision and Artificial Intelligence. Energy Regression of a standalone calorimeter system could be a similar standard for the field of Calorimetry.