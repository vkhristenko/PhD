\section{Conclusions} \label{section:simulations_conclusions}
Two standalone simulations of the potential candidates for the CMS Phase 2 upgrade have been built and examined in this chapter: the High Granularity Calorimeter and the ``Shashlik+HE'' system. The {\sc Geant4} toolkit has been utilized for the purpose of geometry, physics and readout simulation. The baseline performance of both systems has been established. It was observed that Shashlik part has better energy resolution characteristics than the electromagnetic component of the HGC, with asymptotic resolution of 0.6\% vs 1\%. The statistical fluctuations also show a factor of 2 improvement for the Shashlik over the HGC.

Standalone Simulations of Calorimeters are ideal playgrounds for optimizing the parameters of future systems. It avoids unnecessary overhead, which is always introduced once you try to scale things up and use very precise simulations as are used for the CMS detector. The time to physics is substantially reduced and allows to accelerate the development of reconstruction techniques. A very interesting future use case is the Machine Learning (ML) Benchmarking. In comparison, {\sc Mnist} Digit Classification, \cite{mnist}, has become the de facto benchmarking for the field of Computer Vision and Artificial Intelligence. Energy Regression of a standalone calorimeter system could be a similar standard for the field of Calorimetry.