\documentclass{slides}
\usepackage{epsfig}
\usepackage{amsmath}
\usepackage{amssymb}
\usepackage{amsthm}


\title{The Hochschild Homology of Skein Algebras}
\author{Michael McLendon \\ Ph.D. Thesis Defense}
\date{July 10, 2002}

\newcommand{\lcr}{\raisebox{-10pt}{\mbox{}\hspace{1pt}
                  \epsfxsize = 1cm \epsfysize = 1cm
                  \epsfbox{figs/leftcross.eps}\hspace{1pt}\mbox{}}}
\newcommand{\ift}{\raisebox{-10pt}{\mbox{}\hspace{1pt}
                  \epsfxsize = 1cm \epsfysize = 1cm
                  \epsfbox{figs/infinity.eps}\hspace{1pt}\mbox{}}}
\newcommand{\zer}{\raisebox{-10pt}{\mbox{}\hspace{1pt}
                  \epsfxsize = 1cm \epsfysize = 1cm
                  \epsfbox{figs/zero.eps}\hspace{1pt}\mbox{}}}
\newcommand{\ot}{\otimes}
\newcommand{\tr}{\mathrm{tr}}
\newcommand{\bbc}{\mathbb{C}}
\newcommand{\slc}{SL_2(\mathbb{C})}
\newcommand{\torus}{\raisebox{-7pt}{\mbox{}\hspace{-5pt}
                  \includegraphics{figs/small-torus.eps}\hspace{-2pt}\mbox{}}}

\newcommand{\torusL}{\raisebox{-7pt}{\mbox{}\hspace{-5pt}
                  \includegraphics{figs/small-torus-l.eps}\hspace{-2pt}\mbox{}}}

\newcommand{\torusM}{\raisebox{-7pt}{\mbox{}\hspace{-5pt}
                  \includegraphics{figs/small-torus-m.eps}\hspace{-2pt}\mbox{}}}

\newcommand{\torusLM}{\raisebox{-7pt}{\mbox{}\hspace{-5pt}
                  \includegraphics{figs/small-torus-lm.eps}\hspace{-2pt}\mbox{}}}

\newcommand{\torusLMinv}{\raisebox{-7pt}{\mbox{}\hspace{-5pt}
                  \includegraphics{figs/small-torus-lm-inv.eps}\hspace{-2pt}\mbox{}}}
\newcommand{\LittleOneHalf}{{\textstyle \frac{1}{2}}}


\newtheorem{lemma}{Lemma}
\newtheorem{theorem}{Theorem}
\newtheorem{definition}{Definition}
\newtheorem{conjecture}{Conjecture}
\newtheorem{example}{Example}
\newtheorem{remark}{Remark}

\begin{document}

%% Slide 0 %%
\maketitle

%% Slide 1 %%
\begin{slide}
\begin{itemize}
\item background
\item the skein algebra of the torus
	\begin{itemize}
	\item the zeroth homology
	\item the first homology
	\end{itemize}
\item the homology of a Heegaard splitting
	\begin{itemize}
	\item the spectral sequence
	\item a condition for torsion
	\end{itemize}
\item examples:  $S^3$, $S^1 \times S^2$, $\mathbb{RP}^3$
\item future research
\end{itemize}
\end{slide}

%% Slide 2 %%
\begin{slide}
\textbf{part one - background}

An $n$-component \textit{framed link} is the smooth embedding of $n$
disjoint annuli into $S^3$.

Two framed links are equivalent if they are related by ambient isotopy
and the Reidemeister $I^{\prime}$, $II$, and $III$ moves.

Given a crossing in a framed link diagram, a \textit{smoothing} is the
replacement of the crossing by two arcs that do not cross.

A positive smoothing is $\lcr \to \zer$.

A negative smoothing is $\lcr \to \ift$.
\end{slide}

%% Slide 3 %%
\begin{slide}
\begin{center}
    \epsfxsize = 10cm
    \epsfysize = 10cm
    \epsfbox{figs/rmove.eps}
    \put(-240,235){$I$}
    \put(-70,235){$II$}
    \put(-160,65){$III$}

Figure 1:  The three Reidemeister moves
\end{center}
\end{slide}

%% Slide 4 %%
\begin{slide}
\begin{center}
    \epsfxsize = 5cm
    \epsfysize = 5cm
    \epsfbox{figs/r-one-prime.eps}
    \put(-75,80){$I^{\prime}$}

Figure 2:  The Reidemeister $I^{\prime}$ move
\end{center}
\end{slide}

%% Slide 5 %%
\begin{slide}
Let $M$ be an oriented $3$-manifold and
let $\mathcal{L}(M)$ denote the set of equivalence classes of framed links in $M$,
including the empty link, $\phi$.
Let $R=\mathbb{Z}[t,t^{-1}]$ be the ring of Laurent polynomials.

Consider the free module
$R \mathcal{L}(M)$ with basis $\mathcal{L(M)}$.  Define $S(M)$ to
be the smallest subspace of $R \mathcal{L}(M)$ containing
all expressions of the form
$$\lcr-t\zer-t^{-1}\ift$$
and
$$L \sqcup \bigcirc + (t^2 + t^{-2}) L,$$ where $L$ is any framed link
and where the framed links in each expression are identical outside
the regions shown in the diagrams.

The \textit{Kauffman bracket skein module}
$K(M)$ is the quotient $R \mathcal{L}(M)/S(M).$
\end{slide}

%% Slide 6 %%
\begin{slide}
Let $F$ be a compact, orientable surface and let
$I$ be the unit interval.  $K(F \times I)$
has a multiplication that comes from laying one link
on top of the other.  This multiplication makes $K(F \times I)$
into an algebra.

To simplify notation and to emphasize the
algebra structure that comes from $F$ (and not from $F \times I$)
we denote the \textit{skein algebra} of the surface $F$ by
$K(F)$.

\vspace{2cm}

\begin{center}
    \epsfxsize = 10cm
    \epsfysize = 2cm
    \epsfbox{figs/prod-struct.eps}
    \put(-250,23){$\alpha$}
    \put(-200,23){$*$}
    \put(-150,23){$\beta$}
    \put(-100,23){$=$}
    \put(-45,35){$\alpha$}
    \put(-45,10){$\beta$}

Figure 3:  Multiplication in $K(F)$
\end{center}
\end{slide}

%% Slide 7 %%
\begin{slide}
A skein algebra has basis composed of
the simple diagrams on the surface.

\begin{itemize}
\item $K(D^2)$ - any skein is a multiple of $\phi$
\item $K(A) \cong R[z]$, polynomials in one variable,
		$z$ is the core of the annulus
\item $K(T^2) \cong R\{(p,q)_T\}$
\end{itemize}

Let $d = gcd(p,q)$.  Then $$(p,q)_T = T_d \Big( \left( \frac{p}{d}, \frac{q}{d} \right) curve \Big).$$
$T_0(x) = 2$, $T_1(x) = x$, and \\ $T_n(x) = T_{n-1}(x)*T_1(x) - T_{n-2}(x)$.
\end{slide}

%% Slide 8 %%
\begin{slide}
Using the basis $\{(p,q)_T\}$, the multiplication in
$K(T^2)$ is given by
the \textit{product-to-sum} formula.

\[ \hspace{-1cm}
(p,q)_T * (r,s)_T = t^{|^{p~q}_{r~s}|}
(p+r,q+s)_T + t^{-|^{p~q}_{r~s}|}
(p-r, q-s)_T.
\]

\end{slide}

%% Slide 9 %%
\begin{slide}
Let $M$ be a closed, orientable, connected 3-manifold.
Then for some non-negative integer $g$ there exist
genus $g$ handlebodies $H_0$ and $H_1$ such that
$H_0 \cap H_1 = \partial H_0 = \partial H_1 = F$
is a closed, orientable, connected genus $g$ surface
and $H_0 \cup H_1 = M$.
We call these two handlebodies a
\textit{Heegaard splitting} of the manifold.

Note that we can take a neighborhood
of the surface $F$ and think of the Heegaard splitting as breaking
the manifold into $H_0$, $F \times I$, and $H_1$, where $H_0$ is
glued to $F \times \{0\}$ by the identity map
and $H_1$ is glued to $F \times \{1\}$ by a gluing map $f$.
We will model Heegaard splittings in this way, and we will be
interested in the properties of the gluing map $f$.
\end{slide}

%% Slide 10 %%
\begin{slide}
Hochschild homology is a functor that associates an ordered
collection of $R$-modules to an $R$-algebra $A$ and an
$A$-bimodule $B$.

The Hochschild chain complex has chains
$C_n$ given by
\[
C_n = C_n(A; B) = B \ot (A^{\ot n}) = B \ot 
\underbrace{A \ot A \ot \dots \ot A}_{n~\mathrm{times}}
\]
for $n \geq 0$ and
$C_n = 0$ for $n < 0$.
\end{slide}

%% Slide 11 %%
\begin{slide}
The Hochschild boundary map $d_n : C_n \to C_{n-1}$ is given by
\begin{eqnarray}
d_n ( b \ot a_1 \ot \dots \ot a_n )
& = & b a_1 \ot a_2 \ot a_3 \ot \dots \ot a_n \nonumber \\
& & \hspace{-4cm} - b \ot a_1 a_2 \ot a_3 \ot \dots \ot a_n \nonumber \\
& & \hspace{-4cm} + b \ot a_1 \ot a_2 a_3 \ot \dots \ot a_n \nonumber \\
& & \hspace{-4cm} + \dots + (-1)^{n-1} b \ot a_1 \ot \dots \ot a_{n-1} a_n \nonumber \\
& & \hspace{-4cm} + (-1)^{n} a_n b \ot a_1 \ot a_2 \ot \dots \ot a_{n-1} \nonumber
\end{eqnarray}
where the products $a_i a_{i+1}$ take place in the algebra $A$
and the products $b a_1$ and $a_n b$ come from the respective
right and left actions of $A$ on $B$.

The Hochschild homology of $A$ with coefficients in $B$ is the homology
of the Hochschild complex and is denoted $HH_{*}(A; B)$.
If $B = A$ we will denote $HH_{*}(A; A)$ by $HH_{*}(A)$.
\end{slide}

%% Slide 12 %%
\begin{slide}
\textbf{part two - the skein algebra of the torus}

Let $F$ be the
standard 2-torus $T^2$ and let
$t$ be a complex number that is not zero and not a root of unity.

We consider the Hochschild homology
of the specialized skein algebra $A = K_t(T^2)$ with
coefficients in $A$ itself.
The Hochschild complex becomes
\[
\dots \to A \ot A \ot A \to A \ot A \to A \to 0
\]

Recall that $d_1: A \ot A \to A$ is defined by
$$d_1(a \ot b) = ab - ba.$$  The image of $d_1$
in $A$ is the subalgebra of $A$ generated by these
commutators.  Denote this subalgebra by $C(A)$.
The kernel of the zero map $d_0 : A \to 0$ is
all of $A$, thus $$HH_0(A) = A/C(A).$$
\end{slide}


%% Slide 13 %%
\begin{slide}
\begin{lemma}
$(x,y)_T + C(A)$ and $(z,w)_T + C(A)$ are equal
if $x + z$ is even, $y + w$ is
even, and ${|^{x~y}_{z~w}|} \neq 0$.
\label{lemma1}
\end{lemma}

\proof
Suppose we have integers $x$, $y$,
$z$, $w$ such that $x+z$ is even, $y + w$ is even and
${|^{x~y}_{z~w}|} \neq 0$.

Let $p = \frac{x+z}{2}$,
$q = \frac{y+w}{2}$, $r = \frac{x-z}{2}$, and $s = \frac{y-w}{2}$.
Since 
${|^{x~y}_{z~w}|} \neq 0$,
elementary matrix operations lead to
${|^{p~q}_{r~s}|} \neq 0$.
Let $\alpha =
{|^{p~q}_{r~s}|} \neq 0$.

Then 
\begin{eqnarray}
(p,q)_T * (r,s)_T & = & t^{\alpha} (x,y)_T + t^{-\alpha} (z,w)_T \nonumber \\
(r,s)_T * (p,q)_T & = & t^{-\alpha} (x,y)_T + t^{\alpha} (-z,-w)_T \nonumber
\end{eqnarray}
\end{slide}

\begin{slide}
Since orientation doesn't matter in $A$,
$$(-z,-w)_T = (z,w)_T.$$

Therefore
$$(p,q)_T * (r,s)_T - (r,s)_T * (p,q)_T$$
$$ = (t^{\alpha} - t^{-\alpha})((x,y)_T - (z,w)_T).$$

Since $\alpha \neq 0$ and $t$ is not a root of unity, we can divide by
$(t^{\alpha} - t^{-\alpha}) \neq 0$ to get
$$(x,y)_T - (z,w)_T$$
$$= \frac{1}{t^{\alpha} - t^{-\alpha}}
\Big( (p,q)_T * (r,s)_T - (r,s)_T * (p,q)_T \Big) \in C(A).$$

Thus $(x,y)_T + C(A) = (z,w)_T + C(A)$ as cosets in $A/C(A)$.
\qed
\end{slide}

\begin{slide}
%%  I have to decide how much detail I want to put in the slides.
%%  I would like to drastically shorten the proof below, for example.
\begin{theorem}
$HH_0(A) = A/C(A)$ is a five dimensional
vector space over $\mathbb{C}$.
\label{five-dim}
\end{theorem}

\proof
Motivated by Lemma \ref{lemma1}, we define a map
$$\varphi \colon
A \to \mathbb{C} \left\{ \phi, ee, eo, oe, oo \right\}$$
by
$$(p,q)_T \mapsto \left\{
{\begin{array}{ccc}
\phi & {\rm if} & p=0,~q=0 \\
ee & {\rm if} & p~{\rm even},~q~{\rm even} \\
eo & {\rm if} & p~{\rm even},~q~{\rm odd} \\
oe & {\rm if} & p~{\rm odd},~q~{\rm even} \\
oo & {\rm if} & p~{\rm odd},~q~{\rm odd}
\end{array}}
\right.$$
then extend linearly.

Consider
\begin{eqnarray}
c & = & (p,q)_T * (r,s)_T - (r,s)_T * (p,q)_T \nonumber \\
& = & \Big( t^{|^{p~q}_{r~s}|} - t^{- |^{p~q}_{r~s}|} \Big)
\Big( (p+r, q+s)_T - (p-r, q-s)_T \Big). \nonumber
\end{eqnarray}
Then $\varphi(c) = 0$, hence $C(A) \subset {\rm ker}(\varphi)$.
\end{slide}

\begin{slide}

Now take $k \in {\rm ker}(\varphi)$.  Then
\begin{eqnarray}
k & = & \sum_{{\rm finite}} \lambda_{(p,q)} (p,q)_T \nonumber \\
& = & \lambda_{(0,0)} (0,0)_T + \sum_{ee} \lambda_{(p,q)}
(p,q)_T + \nonumber \\
& &  \dots + \sum_{oo} \lambda_{(p,q)} (p,q)_T \nonumber
\label{parity-sum}
\end{eqnarray}

As a model for the other cases, we work the case where
$$k = \sum_{ee} \lambda_{(p,q)} (p,q)_T$$ and so
$$\sum_{ee} \lambda_{(p,q)} = 0.$$
Now, $$\sum_{ee} \lambda_{(p,q)} (p,q)_T$$ is a finite sum
and $(p,q)_T$ are all of even-even parity, and $(p,q) \neq (0,0)$.
\end{slide}

\begin{slide}
Choose integers
$r$ and $s$ such that $(r,s)_T$ is of even-even parity and
$(r,s)$ is linearly independent to each of the $(p,q)$ in
the sum for $k$.
That is, $\frac{s}{r}$ is a rational slope
that is different from the finite number of rational slopes
$\frac{q}{p}$.

Now using Lemma \ref{lemma1} each $(p,q)_T = (r,s)_T$ in
the quotient $A/C(A)$.
Hence
\begin{eqnarray}
k & = & \sum_{ee} \lambda_{(p,q)} (p,q)_T \nonumber \\
& = & \left( \sum_{ee} \lambda_{(p,q)} (r,s)_T \right) + {\rm commutators}
\nonumber \\
& = & \left( \sum_{ee} \lambda_{(p,q)} \right) (r,s)_T + {\rm commutators}
\nonumber \\
& = & {\rm commutators,~since} \sum_{ee} \lambda_{(p,q)} = 0. \nonumber
\end{eqnarray}
Thus $k \in C(A)$.
\end{slide}

\begin{slide}

We could repeat this process for each of the $eo$, $oe$, and $oo$
sums given in Equation \ref{parity-sum}.  Thus for a general
$k \in \mathrm{ker}(\varphi)$, we have $k \in C(A)$.
Hence $\mathrm{ker}(\varphi) \subset C(A)$ and so
$\mathrm{ker}(\varphi) = C(A)$ and
\[
A/C(A) \cong \mathbb{C} \left\{ \phi, ee, eo, oe, oo \right\}.
\]
Thus $HH_0 (A) = A/C(A)$ is a
five dimensional vector space over $\mathbb{C}$.
\qed

A linear functional defined on $A$ that is zero on $C(A)$ is
called a \textit{trace}.  The space of traces is dual to $A / C(A)$.
Thus Theorem \ref{five-dim} implies that there are five traces
on $A = K_t(T^2)$.

There is a trace for each $\mathbb{Z}_2$ homology class of
$T^2$, with one more trace for the empty skein.  Each trace picks off
the coefficients of the basis elements in its corresponding
class.
\end{slide}
%%  I don't think you can use the bibliography environment with
%%  slides.  Maybe I'll just list the bibliography entries seperately,
%%  or just refer people to the thesis.

\begin{slide}
We model $C_1(A) = A \otimes A$ using
weighted, oriented line segments on the plane.
Namely, given an element $(p,q)_T \otimes (r,s)_T \in A \otimes A$
we draw two line segments, one segment from $(p-r,q-s)$ to
$(p+r,q+s)$ and the other from $(-(p-r),-(q-s))$ to
$(-(p+r),-(q+s)$.

We must use two segments because
$(p,q)_T = (-p,-q)_T$ and $(r,s)_T = (-r,-s)_T$ in $A$.

To simplify, we just work with the segments
that lie to the right of the line $y = -x$, keeping in mind
that everything we draw must also be reflected through the origin.

If $\alpha = ps - rq$, then
\begin{eqnarray}
d_1 \Big( (p,q)_T \otimes (r,s)_T \Big) & = & \nonumber \\
& & \hspace{-6cm}(t^\alpha - t^{-\alpha}) \Big( (p+r,q+s)_T - (p-r,q-s)_T \Big) \nonumber
\end{eqnarray}

Hence the boundary of the line segments correspond to the
boundary of $(p,q)_T \otimes (r,s)_T$.

\end{slide}

\begin{slide}

  \begin{center}
    \epsfxsize = 10cm
    \epsfysize = 10cm
    \epsfbox{figs/segment-pair.eps}

Figure 4:  A pair modeling $(3,1)_T \otimes (2,1)_T$
  \end{center}

\end{slide}

\begin{slide}
A 1-chain is a linear combination of these segments.  We can
have additional scalars attached to each segment.  The picture
corresponding to 
a 1-chain is a collection of vertices and weighted, oriented edges
in the plane.

Since $p+r$ and $p-r$ are either both even or both
odd and $q+s$ and $q-s$ are either both even or both odd,
an edge must connect vertices of the same parity.  That is, if an
edge connects $(x_1,y_1)$ to $(x_2,y_2)$, then
$x_1 \equiv x_2~(\mathrm{mod~2})$ and $y_1 \equiv y_2~(\mathrm{mod~2})$.

A 1-cycle in our model is a collection of
vertices and weighted edges such that each vertex
gets a total weight of zero when the weights from each edge
are multiplied by the appropriate $(t^\alpha - t^{-\alpha})$
and added or subtracted as determined by the orientation of the edge. 

Let $\lambda_k = 1 / (t^k - t^{-k})$, for $k = 1, 2, 4, 5$.  Then
a simple $1$-cycle is shown in Figure 5.
\end{slide}

\begin{slide}
  \begin{center}
    \epsfxsize = 10cm
    \epsfysize = 10cm
    \epsfbox{figs/box-cycle.eps}
    \put(0,220){$\lambda_5$}
    \put(-60,190){$-\lambda_2$}
    \put(-60,260){$\lambda_4$}
    \put(-135,220){$-\lambda_1$}
    \put(-300,60){$\lambda_5$}
    \put(-240,90){$-\lambda_2$}
    \put(-240,20){$\lambda_4$}
    \put(-165,60){$-\lambda_1$}

Figure 5:  A simple cycle
  \end{center}
\end{slide}

\begin{slide}
Now consider 1-boundaries.  Start with an element
$(p,q)_T \otimes (r,s)_T \otimes (u,v)_T \in A \otimes A \otimes A$
and consider its boundary under $d_2$.  If
$\alpha = ps - rq$, $\beta = rv-us$, and $\gamma = uq-pv$, then
\begin{eqnarray}
d_2 \Big( (p,q)_T \otimes (r,s)_T \otimes (u,v)_T \Big) & = & \nonumber \\
& & \hspace{-6cm} t^\alpha (p+r,q+s)_T \otimes (u,v)_T \nonumber \\
& & \hspace{-6cm} + t^{-\alpha} (p-r,q-s)_T \otimes (u,v)_T \nonumber \\
& & \hspace{-6cm} - t^\beta (p,q)_T \otimes (r+u,s+v)_T \nonumber \\
& & \hspace{-6cm} - t^{-\beta} (p,q)_T \otimes (r-u,s-v)_T \nonumber \\ 
& & \hspace{-6cm} + t^\gamma (u+p,v+q)_T \otimes (r,s)_T \nonumber \\
& & \hspace{-6cm} + t^{-\gamma} (u-p,v-q)_T \otimes (r,s)_T \nonumber
\end{eqnarray}

In the model, the center of the segment is given by the
first element in the tensor and the slope of the segment is
given by the second element in the tensor.  One can see that
the first two terms in the boundary are parallel (the slope is $v/u$)
and centered at $(p+r,q+s)$ and $(p-r,q-s)$.
\end{slide}

\begin{slide}
Similarly, the
last two terms in the boundary are parallel (the slope is $s/r$)
and centered at $(u+p,v+q)$ and $(u-p,v-q)$.  The middle two
terms in the boundary are both centered at $(p,q)$ with
slopes of $(s+v)/(r+u)$ and $(s-v)/(r-u)$.

Thus a 1-boundary
is described by a parallelogram with its diagonals.
This parallelogram is centered at the point $(p,q)$ on the plane and
has four vertices $(p+r+u,q+s+v)$, $(p+r-u,q+s-v)$,
$(p-r-u,q-s-v)$, and $(p-r+u,q-s+v)$ with the possibility that
some of these vertices must be replaced by their reflection through
the origin in order to get a physical parallelogram.

Figure 6 shows a parallelogram with diagonals
modeling the boundary of $(p,q)_T \ot (r,s)_T \ot (u,v)_T$.
\end{slide}

\begin{slide}
  \begin{center}
    \epsfxsize = 6.67cm
    \epsfysize = 4cm
    \epsfbox{figs/pwd.eps}
    \put(-50,130){$(p+r+u, q+s+v)$}
    \put(-200,130){$(p+r-u, q+s-v)$}
    \put(-100,-20){$(p-r+u, q-s+v)$}
    \put(-250,-20){$(p-r-u, q-s-v)$}
    \put(-120, 10){$t^{-\alpha}$}
    \put(-80, 115){$t^{\alpha}$}
    \put(-140, 40){$t^{\beta}$}
    \put(-120, 90){$t^{-\beta}$}
    \put(-15, 60){$t^{\gamma}$}
    \put(-185, 60){$t^{-\gamma}$}

Figure 6:  The boundary of $(p,q)_T \ot (r,s)_T \ot (u,v)_T$
  \end{center}
\end{slide}

\begin{slide}
Let's take a closer look at the boundary map
\[
d_2(a \otimes b \otimes c) =
ab \otimes c - a \otimes bc + ca \otimes b.
\]
In the quotient $Z_1(A)/B_1(A)$, we
have $a \otimes bc = ab \otimes c + ca \otimes b$.

Using the product-to-sum formula, we can write
\[
(1,1)_T = t(1,0)_T*(0,1)_T - t^{-1}(0,1)_T*(1,0)_T.
\]
\end{slide}

\begin{slide}
In general,
we can write an element $(p,q)_T \in A$ as
\begin{eqnarray}
(p,q)_T & = & \frac{1}{(t^{2p} - t^{-2p})^q} \nonumber \\
& & \hspace{-2cm} \sum_{k=0}^q (-1)^k \binom{q}{k} t^{p(q-2k)}
(0,1)_T^k T_p(1,0) (0,1)_T^{q-k}. \nonumber
\end{eqnarray}
Here $T_p(1,0)= (p,0)_T$ is the the $p$th Chebyshev
polynomial evaluated on the $(1,0)$-curve on the torus.
To prove this we fix $p$ and then use the product-to-sum formula to induct
on $q$.  Of course, a similar formula could be obtained by reversing
the roles of $p$ and $q$.
\end{slide}

\begin{slide}

A useful property of the equation above is that
we can now express $(r,s)_T$ as the sum of products of $(1,0)_T$'s
and $(0,1)_T$'s.  We can then use the formula
$a \otimes bc = ab \otimes c + ca \otimes b$, for $a,b,c \in A$,
to break a product in the second position into two terms.

Thus $(p,q)_T \otimes (r,s)_T = x \otimes (1,0)_T +
y \otimes (0,1)_T$, with $x,y \in A$.  Therefore any
element in $HH_1(A) = Z_1(A)/B_1(A)$ can be written as a
linear combination of horizontal and vertical length 2 segments.

We call such a combination of horizontal and vertical length
2 segments a {\em mesh}.
Figure 7 shows the mesh associated with
the cycle in Figure 5.  The specific weights have
been suppressed in the figure.
\end{slide}

\begin{slide}
  \begin{center}
    \epsfxsize = 10cm
    \epsfysize = 10cm
    \epsfbox{figs/box-cycle-mesh.eps}

Figure 7:  The mesh of the cycle in Figure 5
  \end{center}
\end{slide}

\begin{slide}
\begin{definition}
Let $S_{(p,q)}$ denote the $2 \times 2$ square
centered at $(p,q)$ shown in Figure 8 with
$\lambda_i = 1 / (t^i - t^{-i})$.  In other words,
\begin{eqnarray}
S_{(p,q)} & = & \lambda_{p+1} (p+1,q)_T \ot (0,1)_T
+ \lambda_{q+1}(p,q+1)_T \ot (1,0)_T \nonumber \\
& & - \lambda_{p-1} (p-1,q)_T
\ot (0,1)_T - \lambda_{q-1} (p,q-1)_T \ot (1,0)_T. \nonumber
\end{eqnarray}
We call $S_{(p,q)}$ the \textit{square cycle} centered at
$(p,q)$.  Note that $\lambda_0$ is undefined, so $S_{(p,q)}$
is defined only when $p, q \neq \pm 1$.
\end{definition}

  \begin{center}
    \epsfxsize = 10cm
    \epsfysize = 10cm
    \epsfbox{figs/s_pq.eps}
    \put(-20,220){$\lambda_{p+1}$}
    \put(-120,220){$-\lambda_{p-1}$}
    \put(-80,260){$\lambda_{q+1}$}
    \put(-80,180){$-\lambda_{q-1}$}
    \put(-40,120){$p+1$}
    \put(-100,120){$p-1$}
    \put(-180,260){$q+1$}
    \put(-180,200){$q-1$}

Figure 8:  The square cycle $S_{(p,q)}$
  \end{center}
\end{slide}

\begin{slide}
\begin{lemma}
The square cycle centered at $(p,q)$ is a cycle.  That is,
$d_1(S_{(p,q)}) = 0$.
\end{lemma}

\vspace{1cm}

\begin{lemma}
The set of all $S_{(p,q)}$ together with the set of all length $2$ edges
along the $x$-axis and $y$-axis span $HH_1(A)$.
\label{lemma2}
\end{lemma}

\proof
Given a cycle $z \in Z_1(A)$, represent it with a mesh,
also denoted $z$.
Choose a vertex $P=(x,y)$ in the mesh $z$ such that the
sum $x+y$ is maximal as shown in Figure 9.
$P$ has two incident edges,
one vertical and one horizontal, each of length $2$.
Denote the vertical edge's weight by $\alpha_x$ and the horizontal edge's
weight by $\alpha_y$.  Since $z$ is a cycle, the weight at vertex
$P$ must be $0$ when the boundary is taken.  Thus
\[\alpha_x (t^{x} - t^{-x}) - \alpha_y (t^{-y} - t^{y}) = 0\]
and so
\[\frac{\alpha_x}{\lambda_x} - \frac{\alpha_y}{\lambda_y} = 0.\]
\end{slide}

\begin{slide}
  \begin{center}
    \epsfxsize = 10cm
    \epsfysize = 10cm
    \epsfbox{figs/mesh-proof.eps}
    \put(0,240){$P=(x,y)$}
    \put(0,210){$\alpha_x$}
    \put(-25,240){$\alpha_y$}

Figure 9:  The mesh $z$ with extreme vertex $P$
  \end{center}

\end{slide}

\begin{slide}
Now add a multiple of $S_{(x-1,y-1)}$ to cancel the weights
of the two edges incident at the vertex
$P$.  That is, write
$z$ as $(\alpha_x/\lambda_x) S_{(x-1,y-1)} + z^{\prime}$ where $z^{\prime}$ is
the mesh $z$ with $P$ removed, the edges incident to $P$ removed,
and the weights on the edges centered at $(x - 2, y - 1)$ and
$(x - 1, y - 2)$ adjusted to include the contribution
from $(\alpha_x / \lambda_x) S_{(x-1,y-1)}$.

Since each mesh has a finite number of boxes, we can continue
this process until we are left with no box or with length $2$ edges along the
axes.
\qed
\end{slide}


\begin{slide}

\begin{lemma}
There is a linear combination of $S_{(p,q)}$,
$S_{(p-2,q)}$, $S_{(p,q-2)}$, and $S_{(p-2,q-2)}$ that is
null homologous in $HH_1(A)$.
\label{lemma3}
\end{lemma}

\proof
Consider the boundaries given by
$$\alpha_{(p,q)} = d_2 \Big( (p,q)_T \otimes (1,0)_T
\otimes (0,1)_T \Big)$$
and $$\beta_{(p,q)} = d_2 \Big( (p,q)_T \otimes (0,1)_T
\otimes (1,0)_T \Big).$$  They are composed of
the same six edges, but the weights are slightly different.
In fact, $\alpha_{(p,q)} - t^2 \beta_{(p,q)}$
cancels one of the diagonals and gives the edges shown in Figure
10.

  \begin{center}
    \epsfxsize = 2cm
    \epsfysize = 2cm
    \epsfbox{figs/diag-cancel.eps}

Figure 10:  $\alpha_{(p,q)} - t^2 \beta_{(p,q)}$
  \end{center}
\end{slide}

\begin{slide}
By similar processes on the pairs
$\alpha_{(p-2,q)}$ and $\beta_{(p-2,q)}$,
$\alpha_{(p,q-2)}$ and $\beta_{(p,q-2)}$,
and
$\alpha_{(p-2,q-2)}$ and $\beta_{(p-2,q-2)}$ we can get a boundary
composed of the edges shown in Figure 11.

  \begin{center}
    \epsfxsize = 4cm
    \epsfysize = 4cm
    \epsfbox{figs/four-panes-with-diag.eps}

Figure 11:  Combination of $4$ $\alpha$'s and $4$ $\beta$'s
  \end{center}

Next we use a boundary of the form
$$(p-1,q-1)_T \otimes (1,1)_T \otimes (1,-1)_T$$ to cancel
the four remaining diagonals.  This adds two length $4$
segments, one vertical and one horizontal.  We then
use degenerate boundaries to write each length $4$ segment
as two length $2$ segments.
\end{slide}

\begin{slide}

  \begin{center}
    \epsfxsize = 4cm
    \epsfysize = 4cm
    \epsfbox{figs/four-pane.eps}

Figure 12:  A linear combination of boundaries
  \end{center}

Thus we have a linear combination of boundaries, $C$, composed of the
segments shown in Figure 12.  It is a straightforward, though
tedious, task to keep track of all the coefficients and see that $C$
is the following linear combination of
$S_{(p,q)}$, $S_{(p-2,q)}$, $S_{(p,q-2)}$, and $S_{(p-2,q-2)}$.
\begin{eqnarray}
C & = & t^{-q}(1-t^{2p+2}) (t^{q+1} - t^{-q-1}) S_{(p,q)} + \nonumber \\
& &		t^{-q}(1-t^{-2p+6}) (t^{q+1} - t^{-q-1}) S_{(p-2,q)} + \nonumber \\
& &		t^{p}(t^{2q-4} - t^2) (t^{p+1} - t^{-p-1}) S_{(p, q-2)} - \nonumber \\
& &		t^{q}(t^{-2} - t^{-2p+4}) (t^{q-3} - t^{-q+3}) S_{(p-2,q-2)} \nonumber
\end{eqnarray}
\qed
\end{slide}

% Slide 37
\begin{slide}
\begin{theorem}
As a vector space, $HH_1(A)$ is spanned by length $2$ segments
along the $x$-axis and $y$-axis together with boxes of the form
$S_{(x,0)}$, $S_{(0,y)}$, $S_{(x,3)}$ and $S_{(3,y)}$.
\label{spans}
\end{theorem}

\proof
Using Lemma \ref{lemma2} we can write any cycle
as a combination of $S_{(p,q)}$'s.
Based on the parity of $p$ and $q$, there are four types of
\textit{connected} meshes.

Consider the case where the box cycles are all centered on
$(p,q)$ where both $p$ and $q$ are even.  Thus the vertices of
the mesh have odd $x$ and $y$ coordinates.
Using Lemma \ref{lemma3}
we can add boundaries to remove outermost boxes and push
the mesh toward the axes.  Eventually the mesh will
reduce to boxes centered along the two axes as shown in
Figure 13.  Each of these boxes is of the
form $S_{(x,0)}$ or $S_{(0,y)}$.
\end{slide}

\begin{slide}
In each of the other cases we can also apply Lemma \ref{lemma3} to push the
mesh toward the axes.  When the box cycles are centered
on even $p$ and odd $q$, the mesh reduces to Figure
14.  For odd $p$ and even $q$, the mesh
reduces to Figure 15.  Finally, for
odd $p$ and odd $q$, the mesh reduces to Figure 16.

Thus in each case, the mesh reduces to some linear combination of
length $2$ segments along the axes and boxes of the form $S_{(x,0)}$,
$S_{(0,y)}$, $S_{(x,3)}$, and $S_{(3,y)}$.
\qed
\end{slide}

\begin{slide}
  \begin{center}
    \epsfxsize = 6cm
    \epsfysize = 6cm
    \epsfbox{figs/axial-boxes-case1.eps}

Figure 13:  Reduced mesh, $p$ even, $q$ even
  \end{center}

  \begin{center}
    \epsfxsize = 6cm
    \epsfysize = 6cm
    \epsfbox{figs/axial-boxes-case2.eps}

Figure 14:  Reduced mesh, $p$ even, $q$ odd
  \end{center}
\end{slide}

\begin{slide}
  \begin{center}
    \epsfxsize = 6cm
    \epsfysize = 6cm
    \epsfbox{figs/axial-boxes-case3.eps}

Figure 15:  Reduced mesh, $p$ odd, $q$ even
  \end{center}

  \begin{center}
    \epsfxsize = 6cm
    \epsfysize = 6cm
    \epsfbox{figs/axial-boxes-case4.eps}

Figure 16:  Reduced mesh, $p$ odd, $q$ odd
  \end{center}

\end{slide}

\begin{slide}
\textbf{part three - the hochschild homology of a heegaard splitting}

Next we look at $A=K(F)$, $B_0=K(H_0)$,
and $B_1=K(H_1)$ and compute
$HH_{*}(A; B_0 \ot B_1)$.

We do this because
$$HH_0(A; B_0 \ot B_1) \cong B_1 \ot_{A} B_0 \cong K(M).$$
We construct a spectral sequence that uses properties of $K_{-1}$ to determine
if torsion exists in $K(M)$.

Hoste and Przytycki have shown that
the skein module of each lens space other than $S^1 \times S^2$ is free
and that the skein module of $S^1 \times S^2$ has torsion
elements.
\end{slide}

\begin{slide}
\begin{lemma}(Hoste-Przytycki)
Consider the manifold $F \times [0,1]$ where $F$ is a surface.
Let $\alpha$ be a simple closed curve on $F \times \{ 0 \}$ (or $F \times \{ 1 \}$).
Let $H$ be the manifold obtained by attaching a $2$-handle to $\alpha$.
Let $I$ be the submodule of $K(F)$ generated by relations of the form
$\{ s - h(s) \}$ where $s \in K(F)$ is a skein and $h(s)$ is the skein $s$
modified by a handleslide across $\alpha$.  Then $K(H) = K(F)/I$.
\label{handleslide}
\end{lemma}

The lemma above generalizes to the case where the manifold is obtained by
attaching more than one $2$-handle to $F \times [0,1]$.  Say we attach
a $2$-handle to $F \times \{ 0 \}$ along $\alpha$ and attach another
$2$-handle to $F \times \{ 1 \}$ along $\beta$.  Call the resulting
manifold $M$.

If $I$ is the submodule
of $K(F)$ generated by handleslides along $\alpha$ and $J$ is the submodule
of $K(F)$ generated by handleslides along $\beta$, then
$K(M) = K(F)/(I+J)$.
\end{slide}

\begin{slide}
\begin{theorem}
Let $M$ be a closed, connected, oriented $3$-manifold with Heegaard splitting
$M= H_0 \cup H_1$, $F = H_0 \cap H_1$.  Then
$$K(M) = K(H_1) \ot_{K(F)} K(H_0).$$
\label{kmhs}
\end{theorem}

\proof
$H_0$ is obtained from
$F \times [0,1]$ by attaching $2$-handles to $F \times [0,1]$
along attaching curves $\alpha_k$.  Likewise, $H_1$ is obtained
from $F \times [0,1]$ by attaching $2$-handles to along curves $\beta_n$.
For $i \in \{0,1\}$, let $S_i$ be the submodule of $K(F)$
generated by handleslides across the $\alpha_k$ or across the $\beta_n$, respectively.
We can apply Lemma \ref{handleslide} to each $H_i$ and to $M$.  Then
we have $K(H_i) = K(F)/S_i$ and $K(M) = K(F)/(S_1 + S_0)$.
\end{slide}

\begin{slide}
We know that $A/I \ot_A B \cong B/(IB)$ from homological algebra.
Let $A = K(F)$.  Consider $A/S_1 \ot_A A/S_0 \cong \frac{A/S_0}{S_1 (A/S_0)}$.
An element from $S_1 (A/S_0)$ looks like $sa + S_0$ where $s \in S_1$
and $a \in A$.
An element of $\frac{A/S_0}{S_1 (A/S_0)}$ looks like
$(a^{\prime} + S_0) + (sa + S_0)$.
Recall that the empty skein $\phi$ is the multiplicative
identity in $A$, thus $sa$ runs over all of $S_1$ and so
$(a^{\prime} + S_0) + (sa + S_0) = a^{\prime} + (S_1 + S_0)$.
Thus $K(H_1) \ot_{K(F)} K(H_0) \cong A/S_1 \ot_A A/S_0
\cong \frac{A/S_0}{S_1 (A/S_0)} \cong A/(S_1 + S_0) \cong K(M)$.
\qed
\end{slide}

\begin{slide}

Another interesting and useful property of the skein module $K(M)$ comes
when we specialize at $t=-1$.  The coordinate ring of
the $SL_2(\mathbb{C})$-characters on $\pi_1(M)$
is a quotient of this specialization.
This approach has been 
developed by Bullock and also by
Przytycki and Sikora.

Denote the specialization of $K(M)$ at $t=-1$ by $K_{-1}(M)$
and denote the space of $SL_2(\mathbb{C})$-characters by $X(M)$.
Let $\mathbb{C}^{X(M)}$ denote the algebra of functions on
$X(M)$ and let $R(M)$ denote the coordinate ring of $X(M)$.

It is a theorem of Culler and Shalen that
$X(M)$ is an affine algebraic set.  Hence one can consider
the ring of polynomial functions on $X(M)$.  This ring of polynomial
functions is called the \textit{coordinate ring} of $X(M)$.
Indeed, Culler and Shalen show that the coordinate ring is
finitely generated.
% see Prop 1.4.1, p. 116 and Cor. 1.4.5 , p. 119 of Culler and Shalen
\end{slide}

\begin{slide}
An oriented knot in $M$ determines a conjugacy class in $\pi_1(M)$
and thus an oriented knot $L$ defines a function
$\varphi_L : X(M) \to \mathbb{C}$ by $\varphi_L (\chi_{\rho}) = \chi_{\rho}(L) = \tr(\rho(L))$
where $\chi_{\rho}$ is the character induced by the representation $\rho$
and the knot $L$ is seen as an element of $\pi_1(M)$.
Since $\tr(A) = \tr(A^{-1})$ for any matrix $A$, the particular orientation on
the knot $L$ is irrelevant.

Let $\mathbb{C}\mathcal{L}(M)$ denote the vector space of
framed links in $M$.  Define a function
$\tilde{\Phi} : \mathbb{C}\mathcal{L}(M) \to \mathbb{C}^{X(M)}$ by
$\tilde{\Phi}(L) = - \varphi_L$ for a knot $L$ and
$\tilde{\Phi}(L) = \prod_i (-\varphi_{L_i})$ for a link
$L$ with components $L_i$.
\end{slide}

\begin{slide}
\begin{theorem}(Bullock)
The map $\tilde{\Phi}$ descends to a map
$\Phi : K_{-1}(M) \to \mathbb{C}^{X(M)}$.  Its
image is the coordinate ring $R(M) \subset \mathbb{C}^{X(M)}$
and its kernel is the nilradical of $K_{-1}(M)$.
\label{bullock}
\end{theorem}

The proof that the map descends follows from the observation that
the skein relation maps to the $SL_2(\mathbb{C})$ trace identity
$\mathrm{tr}(AB) + \mathrm{tr}(AB^{-1}) = \mathrm{tr}(A)\mathrm{tr}(B)$.

Przytycki and Sikora have shown that
the nilradical of $K_{-1}(M)$ is trivial for surfaces and handlebodies.
Thus for surfaces and handlebodies $\Phi$ is an isomorphism between the specialized
skein module and the coordinate ring of the character variety.
\end{slide}

\begin{slide}
\begin{example}
As an example, let's look at the $3$-manifold $M = T^2 \times I$.  We
know that 
$\pi_1(M) = \langle \ell, m~|~\ell m \ell^{-1} m^{-1} = 1 \rangle$.
The coordinate ring $R(M)$ is generated by
$x = -\mathrm{tr}(\rho(m))$, $y = - \mathrm{tr}(\rho(\ell))$,
and $z = -\mathrm{tr}(\rho(\ell m))$, and it has one
relation induced by $\mathrm{tr}(\ell m \ell^{-1} m^{-1}) = 2$.
Hence
\[
R(M) \cong \bbc[x,y,z] / (x^2 + y^2 + z^2 + xyz - 4).
\]
\label{ex-torus}
\end{example}
\end{slide}

\begin{slide}
Now we can use the connection between the specialized skein
module and the coordinate ring in the context of a Heegaard
splitting.

Let $M$ be a 3-manifold with Heegaard
splitting $M = H_0 \cup F \times [0,1] \cup H_1$
and with gluing maps $f_0 : H_0 \to F \times \{0\}$ and
$f_1: H_1 \to F \times \{1\}$.
We will take $f_0$ to be the identity, hence the structure of the manifold
is described by $f_1$.

The action of $K(F)$ on $K(H_i)$ is
given by pushing the skeins from $F \times I$ into $H_i$ using the inverse of
the $f_i$ gluing map.
The action of $\alpha, \beta \in K(F)$ on $h_0 \in K(H_0)$ is a left action,
$(\alpha \beta) * h_0 = \alpha * (\beta * h_0)$,
as shown in Figure 17.

The action of $\alpha, \beta \in K(F)$ on $h_1 \in K(H_1)$ is a right action,
$h_1 * (\alpha \beta) = (h_1 * \alpha) * \beta$, as shown in Figure 18.
\end{slide}

\begin{slide}
  \begin{center}
    \epsfxsize = 8cm
    \epsfysize = 4.5cm
    \epsfbox{figs/left-right-action.eps}
    \put(-140,70){$\alpha$}
    \put(-95,70){$\beta$}
    \put(-45,70){$h_0$}
    \put(-150,0){$F \times I$}
    \put(-70,0){$H_0$}
    \put(-210,0){$H_1$}

Figure 17:  $(\alpha \beta) * h_0 = \alpha * (\beta * h_0)$ defines a left action
  \end{center}

\vspace{2cm}

  \begin{center}
    \epsfxsize = 8cm
    \epsfysize = 4.5cm
    \epsfbox{figs/left-right-action.eps}
    \put(-140,70){$\alpha$}
    \put(-95,70){$\beta$}
    \put(-190,70){$h_1$}
    \put(-150,0){$F \times I$}
    \put(-70,0){$H_0$}
    \put(-210,0){$H_1$}

Figure 18:  $h_1 * (\alpha \beta) = (h_1 * \alpha) * \beta$ defines a right action
  \end{center}
\end{slide}

\begin{slide}

For $R = \mathbb{Z}[t, t^{-1}]$, we know that $A = K(F)$ is an algebra over
$R$.  Also, $B_0 = K(H_0)$,
and $B_1 = K(H_1)$ are modules over $R$.  Since $B_0$ is a
left $A$-module and $B_1$ is a right $A$-module, the tensor product
$B = B_0 \ot_R B_1$ is a bimodule over $A$.
We will use the unspecified tensor $\ot$ to denote $\ot_R$.

Now we look at $HH_{*}(A; B_0 \ot B_1)$.
With the choice of $B = B_0 \ot B_1$, the chains become
\[
C_n(A; B_0 \ot B_1) = (B_0 \ot B_1) \ot (A^{\ot n})
\]
for $n \geq 0$ and
$C_n = 0$ for $n < 0$.

Rearrange the terms in the tensor product
so that the $C_n$ become
\[
C_n(A; B_0 \ot B_1) = B_1 \ot
\underbrace{A \ot A \ot \dots \ot A}_{n~\mathrm{times}} \ot B_0.
\]
\end{slide}

\begin{slide}
Then $d_n$ is
\begin{eqnarray}
d_n ( b_1 \ot a_1 \ot \dots \ot a_n \ot b_0 ) & = & \nonumber \\
& & \hspace{-5cm} b_1 a_1 \ot a_2 \ot a_3 \ot \dots \ot a_n \ot b_0 \nonumber \\
& & \hspace{-5cm} - b_1 \ot a_1 a_2 \ot a_3 \ot \dots \ot a_n \ot b_0 \nonumber \\
& & \hspace{-5cm} + \dots + (-1)^{n} b_1 \ot a_1 \ot \dots \ot a_n b_0. \nonumber
\end{eqnarray}

Notice that without the $B_1$, the sequence
$$\dots \to A \ot A \ot B_0 \to
A \ot B_0 \to B_0 \to 0$$ is a free (hence projective) resolution of $B_0$.
If we delete $B_0$ from this complex, tensor over $A$ on the left by $B_1$
and compute the homology, we get $\mathrm{Tor}_i^A(B_1, B_0)$.
That is, $\mathrm{Tor}_i^A(B_1,B_0)$ is the homology of the following complex
$$\dots \to B_1 \ot_A
A \ot A \ot B_0 \to B_1 \ot_A A \ot B_0 \to 0.$$
Since $B_1 \ot_A A = B_1$, the complex becomes
$$\dots \to B_1 \ot A \ot B_0
\to B_1 \ot B_0 \to 0.$$
Therefore the $\mathrm{Tor}$ complex is exactly the same as the
Hochschild complex, that is,
\[HH_i(A; B_0 \ot B_1) = \mathrm{Tor}_i^A(B_1, B_0).\]
\end{slide}

\begin{slide}
\begin{lemma}
The zeroth Hochschild homology of a Heegaard splitting
is the skein module of the 3-manifold $M$.
\end{lemma}

\proof
$\mathrm{Tor}_0$ corresponds to $\ot$, thus
$$\mathrm{Tor}_0^A(B_1, B_0) = B_1 \ot_A B_0.$$
We know from Theorem \ref{kmhs} that $$K(M) = K(H_1) \ot_{K(F)} K(H_0),$$ thus
\begin{eqnarray}
K(M) & = & B_1 \ot_A B_0 \nonumber \\
& = & \mathrm{Tor}_0^A(B_1, B_0) \nonumber \\
& = & HH_0(A; B_0 \ot B_1). \nonumber
\end{eqnarray}
\qed
\end{slide}

\begin{slide}
Next we use a filtration on $R$, $A$, $B_0$, and $B_1$ to get a
spectral sequence and compute torsion in $K(M)$.  We will follow
a process used by Brylinski to study Poisson
manifolds.

The ring $R = \mathbb{Z}[t,t^{-1}]$ of Laurent polynomials has a
filtration by the ideals corresponding to powers of $(1+t)$.
\[\dots \subset (1+t)^3 R \subset (1+t)^2 R \subset (1+t) R \subset R\]
This is a decreasing filtration.  By manipulating the indices, we can
use this filtration to get an increasing filtration.  In particular, if
$$\mathcal{F}_s(R) = (1+t)^{-s} R$$ for $s \leq 0$ and
$$\mathcal{F}_s(R) = R$$ for $s > 0$, then $\mathcal{F}$ is
an increasing filtration on $R$.
\end{slide}

\begin{slide}

This filtration extends to the $R$-modules $A=K(F)$ and $B_i=K(H_i)$ by
\[\dots \subset (1+t)^3 A \subset (1+t)^2 A \subset (1+t) A \subset A\] and
\[\dots \subset (1+t)^3 B_i \subset (1+t)^2 B_i \subset (1+t) B_i \subset B_i.\]
It also extends to the Hochschild complex $C_n = C_n(A;B_0 \ot B_1)$ by the
following from Brylinski.
\begin{eqnarray}
\mathcal{F}_s(C_n) & = & \mathcal{F}_s \left( B_1 \ot (A^{\ot n})
                         \ot B_0 \right) \nonumber \\
& & \hspace{-4cm} = \sum_{s_0 + \dots + s_{n+1} \leq s} \left( \mathcal{F}_{s_0}(B_1) \ot
      \dots \ot 
      \mathcal{F}_{s_n}(A) \ot \mathcal{F}_{s_{n+1}}(B_0) \right) \nonumber \\
& & \hspace{-4cm} = \sum_{\sum s_i \leq s} (1+t)^{-\sum s_i} \left( B_1 \ot (A^{\ot n})
      \ot B_0 \right) \nonumber
\end{eqnarray}
\end{slide}

\begin{slide}
Now we create a spectral sequence $\{ E^r \}$ beginning at the
$E^0$ level with
\[E^0_{p,q} = \mathcal{F}_{p}(C_{p+q}) / \mathcal{F}_{p-1}(C_{p+q})\]
and the $E^0$ level boundaries $\Delta^0_n$ are given by the Hochschild boundary
map $d_n$.

We move from the $E^0$ level to the $E^1$ level by taking
homology and letting the new boundary map $\Delta^1_n$ be the
connecting homomorphism induced by the short exact sequence
\[
0 \to \frac{\mathcal{F}_{p-1}(C_n)}{\mathcal{F}_{p-2}(C_n)} \to
\frac{\mathcal{F}_{p}(C_n)}{\mathcal{F}_{p-2}(C_n)} \to
\frac{\mathcal{F}_{p}(C_n)}{\mathcal{F}_{p-1}(C_n)} \to 0.
\]

In general,
 $E^r_{p,q} = H(E^{r-1}_{p,q}) = \mathrm{ker}(\Delta^{r-1})/\mathrm{im}(\Delta^{r-1})$ and
we set
\[
E^{\infty}_{p,q} = \varinjlim E^r_{p,q}.
\]
\end{slide}

\begin{slide}
The terms at all levels are only nonzero in the second quadrant above
the line $y=-x$.  The $E^0$ level is shown in Figure 19.
We will be particularly concerned with the terms $E^r_{p,-p}$
along the lower diagonal.

  \begin{center}
    \epsfxsize = 10cm
    \epsfysize = 10cm
    \epsfbox{figs/E0-level.eps}
    \put(-50,35){$\frac{B_1 \ot B_0}{(1+t)(B_1 \ot B_0)}$}
    \put(-50,100){$\frac{B_1 \ot A \ot B_0}{(1+t)(B_1 \ot A \ot B_0)}$}
    \put(-50,155){$\frac{B_1 \ot A \ot A \ot B_0}{(1+t)(B_1 \ot A \ot A \ot B_0)}$}
    \put(-50,210){$\frac{B_1 \ot A^{\ot 3} \ot B_0}{(1+t)(B_1 \ot A^{\ot 3} \ot B_0)}$}
    \put(-160,90){$\frac{(1+t)(B_1 \ot B_0)}{(1+t)^2(B_1 \ot B_0)}$}
    \put(-220,145){$\frac{(1+t)^2(B_1 \ot B_0)}{(1+t)^3(B_1 \ot B_0)}$}
    \put(-280,200){$\frac{(1+t)^3(B_1 \ot B_0)}{(1+t)^4(B_1 \ot B_0)}$}

Figure 19:  The $E^0$ level of the spectral seq.
  \end{center}
\end{slide}

\begin{slide}
Modding out by the ideal generated by $(1+t)$ is the same as setting $(1+t)=0$
or simply evaluating the polynomials at $t=-1$.
Since evaluating at $t=-1$ yields the specialized skein module,
we have, for example, $$K(F)/(1+t)K(F) \cong K_{-1}(F).$$

These quotients are consistent with the tensor product, thus,
\[
\frac{B_1 \ot (A^{\ot n}) \ot B_0}{(1+t) (B_1 \ot (A^{\ot n}) \ot B_0)}
\]
is isomorphic to
\[K_{-1}(H_1) \ot (K_{-1}(F))^{\ot n} \ot K_{-1}(H_0).
\]
\end{slide}

\begin{slide}
In addition, for $X \in \{A, B_0, B_1\}$ there is a natural map
from $X$ to $(1+t)^n X$ given by multiplication
by $(1+t)^n$.  Each of these skein modules is free on simple diagrams,
so this induces an isomorphism
\begin{equation}
\frac{X}{(1+t)X} \cong \frac{(1+t)^n X}{(1+t)^{n+1} X}
\label{eq-n-mult}
\end{equation}
and thus every term of the $E^0$ level (in any column)
is isomorphic to the tensor product of specialized skein modules.
\end{slide}

\begin{slide}
Now we want to move from the $E^0$ level to the $E^1$ level.
The terms of the $E^1$ level complexes are the homology modules
of the vertical complexes of the $E^0$ level.  The boundary
maps for the $E^1$ level are the induced connecting homomorphisms
$\Delta^1_n : HH_n \to HH_{n-1}$.  The $E^1$ level is shown in
Figure 20.

  \begin{center}
    \epsfxsize = 10cm
    \epsfysize = 10cm
    \epsfbox{figs/E1-level.eps}
    \put(-55,45){$HH_0$}
    \put(-55,100){$HH_1$}
    \put(-55,155){$HH_2$}
    \put(-55,210){$HH_3$}
    \put(-90, 120){$\Delta^1_1$}
    \put(-90, 177){$\Delta^1_2$}
    \put(-90, 232){$\Delta^1_3$}
    \put(-120,100){$HH_0$}
    \put(-120,155){$HH_1$}
    \put(-120,210){$HH_2$}
    \put(-150, 177){$\Delta^1_1$}
    \put(-150, 232){$\Delta^1_2$}
    \put(-180,155){$HH_0$}
    \put(-180,210){$HH_1$}
    \put(-210, 232){$\Delta^1_1$}
    \put(-230,210){$HH_0$}

Figure 20:  The $E^1$ level of the spectral seq.
  \end{center}
\end{slide}

\begin{slide}

The connecting homomorphisms in this context are essentially the same
as the boundary map, we are just considering the image in a different
quotient space.

For example, a $1$-cycle in $E^0_{0,1}$ is an element
$\alpha \in (B_1 \ot A \ot B_0) / (1+t) (B_1 \ot A \ot B_0)$ that maps to
zero in $(B_1 \ot B_0) / (1+t)(B_1 \ot B_0)$, so $d_1(\alpha)$ is
divisible by $(1+t)$.  The connecting homomorphism $\Delta$ is defined
by $\Delta(\alpha) = d_1(\alpha)$, then consider $d_1(\alpha)$
in $(1+t)(B_1 \ot B_0) / (1+t)^2 (B_1 \ot B_0)$.
For example, $\Delta(\alpha)$ will be zero if $d_1(\alpha)$ is
also divisible by $(1+t)^2$.

Thus, $\Delta^1_1 : HH_1 \to HH_0$ will be the zero map if every element
whose boundary is divisible by $(1+t)$ also has its boundary divisible
by $(1+t)^2$.
\end{slide}

\begin{slide}
When we move to the next level, we will be looking at something
whose boundary is divisible by $(1+t)^2$ and it will be zero under
the next connecting homomorphism if its boundary is also divisible
by $(1+t)^3$.

Thus, all the connecting homomorphisms will be the zero map if every
element whose boundary is divisible by $(1+t)$ also has a boundary
that is divisible by $(1+t)^r$ for all $r$.  This will happen if
the boundary in question is exactly zero, not just some
polynomial divisible by $(1+t)$.

Again, consider Figure 20 and 
notice that these $\Delta$ maps are
horizontal and pointing along the negative $x$ axis.
Also, since our filtered complex (the $E^0$ level)
can be seen as a complex of specialized skein modules, the
$E^1_{p,-p}$ terms are just the Hochschild homology of these
specialized skein modules.
\end{slide}

\begin{slide}
Namely,
\begin{eqnarray}
E^1_{p,-p}
& = & HH_0 \left( K_{-1}(F) ; K_{-1}(H_0) \ot K_{-1}(H_1) \right) \nonumber \\
& = & K_{-1}(H_1) \ot_{K_{-1}(F)} K_{-1}(H_0) \nonumber \\
& = & K_{-1}(M). \nonumber
\end{eqnarray}

We have constructed a spectral sequence from the
filtered Hochschild complex.  We now want to use this
construction to look for torsion in the skein module $K(M)$.

\vspace{2cm}

\begin{definition}
A module $X$ over a ring $R$ has \textit{torsion} if there
exist nonzero elements $r \in R$ and $x \in X$ such that
$rx = 0$.
\end{definition}
\end{slide}

\begin{slide}
\begin{definition}
Let $M$ be a manifold. Recall that $K(M)$, the skein module
of $M$, is a module over the Laurent polynomials
$R=\mathbb{Z}[t,t^{-1}]$.
The filtration $\mathcal{F}_s(R) = (1+t)^{-s} R$ for $s \leq 0$
extends to $K(M)$ as
\[
\dots \subset (1+t)^2 K(M) \subset (1+t) K(M)
\subset K(M).
\]
The quotients $K(M) / (1+t)^n K(M)$ together with the projections
$\theta_n : K(M) / (1+t)^n K(M) \to K(M) / (1+t)^{n-1} K(M)$ form an
injective system.
The \textit{completion} of $K(M)$ is the inverse limit of this system,
\[
\overline{K(M)} = \varprojlim K(M) / (1+t)^n K(M).
\]
The completion can also be seen as the module of all sequences
$\{ a_n \}_{n=0}^{\infty}$ with $a_n \in K(M) / (1+t)^n K(M)$ and $\theta_n(a_n) = a_{n-1}$.
There is a homomorphism $\varphi : K(M) \to \overline{K(M)}$ where
$\varphi ( \alpha)$ is the constant sequence $\{ \alpha \}_{n=0}^{\infty}$.
Note that the kernel of this homomorphism is
\[
\mathrm{Ker}(\varphi) = \bigcap_{n=0}^{\infty}(1+t)^n K(M).
\]
\end{definition}
\end{slide}

% slide 65
\begin{slide}
\begin{theorem}
If the $\Delta^r_1: E^r_{p+1,-p} \to E^r_{p,-p}$ maps are identically zero
at every level in the spectral sequence, then there is
no torsion in $\overline{K(M)}$.
\label{thm-zero-maps}
\end{theorem}

\proof
We focus our attention in the spectral sequence to the terms
that lie along the lower diagonal (the line $y=-x$).  These
are the terms $E^r_{p,-p}$ for $p \leq 0$.

At the $E^1$ level, we have
\[
E^1_{p,-p} = K_{-1}(M).
\]

The maps $\Delta^1_1 : E^1_{p+1,-p} \to E^1_{p,-p}$ are zero
maps, thus
\[E^2_{p,-p} =
\frac{\mathrm{ker}(E^1_{p,-p} \to 0)} {\mathrm{im}(\Delta^1)} = 
E^1_{p,-p}.
\]
That is, the term at position $(p,-p)$ remains unchanged
when we move from the $E^1$ level to the $E^2$ level.
\end{slide}

\begin{slide}

The argument is the same for any $r$.
The maps $\Delta^r_1 : E^r_{p+r,-p-r+1} \to E^r_{p,-p}$ are zero
maps, thus
\[E^{r+1}_{p,-p} =
\frac{\mathrm{ker}(E^r_{p,-p} \to 0)} {\mathrm{im}(\Delta^r)} = 
E^r_{p,-p}.
\]

Thus after the $E^1$ level, the terms $E^r_{p,-p}$ along the lower
diagonal are always just the zeroth Hochschild homology of the
filtered complexes.  In the limit we have
\[
E^{\infty}_{p,-p} = \varinjlim E^r_{p,-p} = E^1_{p,-p} = K_{-1}(M).
\]

The $E^{\infty}$ level of the spectral sequence is shown in Figure 21.
\end{slide}

\begin{slide}
  \begin{center}
    \epsfxsize = 10cm
    \epsfysize = 10cm
    \epsfbox{figs/Einf-level.eps}
    \put(-55,45){$K_{-1}(M)$}
    \put(-120,100){$K_{-1}(M)$}
    \put(-180,155){$K_{-1}(M)$}
    \put(-230,210){$K_{-1}(M)$}

Figure 21:  The $E^{\infty}$ level of the spectral seq.
  \end{center}
\end{slide}

\begin{slide}
Each term along the lower diagonal is isomorphic to $K_{-1}(M)$.
Specifically,
\[
E^{\infty}_{p,-p} \cong \frac{(1+t)^{-p} K(M)}{(1+t)^{-p+1}K(M)},
\]
and the map
\[ \mu_n : \frac{K(M)}{(1+t)K(M)} \to \frac{(1+t)^nK(M)}{(1+t)^{n+1}K(M)} \]
given as multiplication by $(1+t)^n$ is an isomorphism for all $n \geq 1$.
\end{slide}

\begin{slide}
Suppose there exists an $\alpha \in K(M)$ such that $(1+t) \alpha = 0$.
Then $\mu_1(\alpha) = 0$.
Since $\mu_1$ is an isomorphism, $\alpha$ must be zero in $K(M) / (1+t) K(M)$.
Hence $\alpha$ is divisible by $(1+t)$.
Let $\alpha_1 = \alpha / (1+t)$.  Then
$\mu_2(\alpha_1) = (1+t)^2 \alpha = 0$.  Since $\mu_2$ is an isomorphism, $\alpha_1$
must be zero in $K(M) / (1+t) K(M)$.  Hence $\alpha_1$ is divisible by $(1+t)$ and so
$\alpha$ is divisible by $(1+t)^2$.  An induction argument shows that
$\alpha$ is divisible by $(1+t)^n$ for all $n$.  Thus
$\alpha \in \cap_{n=0}^{\infty} (1+t)^n K(M)$
and so $\alpha = 0$ in the completion $\overline{K(M)}$.
Therefore there is no $(1+t)$ torsion in $\overline{K(M)}$.
\end{slide}

\begin{slide}
Lastly, we know that the absence of $(1+t)$ torsion in
$\overline{K(M)}$ implies the absence of torsion in $\overline{K(M)}$.  This
follows from the fact that the specialized skein module
$K_{-1}(M)$ is a vector space, hence is torsion free.
If $f(t)*s = 0$ for a nonzero skein $s$ and a 
polynomial $f(t)$ nonzero and not divisible by $(1+t)$ we
have $f(-1)*s = 0$ for a nonzero number $f(-1)$ and a
nonzero skein $s$.  However, we cannot have such torsion
in $K_{-1}(M)$ because the specialized skein module is
a vector space over $\bbc$.
\qed
\end{slide}

\begin{slide}
\textbf{part four - examples using the spectral sequence}

Now we look at some specific manifolds and
determine when Theorem \ref{thm-zero-maps} may be applied.
We will consider the manifolds $S^3$, $S^1 \times S^2$, and
$\mathbb{RP}^3$ as lens spaces.

A lens space is the result of
gluing two solid tori together by an
orientation reversing homeomorphism of one boundary torus to
the other.  An orientation reversing  homeomorphism
of the torus to itself is determined by the image of the meridian
and the image of the longitude.
These homeomorphisms can be identified with the set of
two by two matrices with integer entries and a determinant
of negative one.
\end{slide}

\begin{slide}
Let the column vectors
%$\ell = (1~0)^T$ and $m = (0~1)^T$
$\ell = \binom{1}{0}$ and $m = \binom{0}{1}$
denote the longitude and meridian, respectively.
For $p, q \in \mathbb{Z}$
with $gcd(p,q)=1$, let the column vector
%$(p~q)^T$
$\binom{p}{q}$
denote the
curve on the torus that traces out $p$
longitudes and $q$ meridians.

Let $\varphi$ be a homeomorphism of
the torus to itself.  The action of $\varphi$ on the torus
is given by an invertible integer matrix applied
to the $\ell$ and $m$ vectors above.
For example
$$
\binom{a~~b}{c~~d}
\binom{1}{0}
=
\binom{a}{c}
$$
and
$$
\binom{a~~b}{c~~d}
\binom{0}{1}
=
\binom{b}{d}
$$
give the action of a matrix on $\ell$ and $m$.  This matrix corresponds to
the map $\varphi$ such that $\varphi(\ell) = a \ell + c m$ and
$\varphi(m) = b \ell + d m$.
\end{slide}

\begin{slide}
In general,
$$
\binom{a~~b}{c~~d}
\binom{p}{q}
=
\binom{ap+bq}{cp+dq}
$$
gives the action of the matrix on the $(p,q)$-curve.

The $L(p,q)$ lens space is obtained by gluing the meridian of
one solid torus to the $(p,q)$-curve on the
other solid torus.  The homeomorphism used for this
gluing must also send the longitude to some curve that intersects
the $(p,q)$-curve exactly once.

There is a
choice as to which curve the longitude will map to,
but the resulting lens space is independent of this choice.
We will use the matrix
$$\binom{x~~p}{y~~q}$$
to denote the gluing homeomorphism for the $L(p,q)$ lens space,
where $x$ and $y$ are
chosen so that $xq - yp = -1$.
\end{slide}

\begin{slide}
Sometimes we will use multiplicative
notation for elements in $\pi_1(T^2)$, e.g.\ $\ell m$ instead
of $\ell + m$ for the $(1,1)$-curve.

Take the genus one Heegaard splitting
$$H_1 \cup_{f_1} T^2 \times I \cup_{f_0} H_0$$ for $S^3$ where
$f_0 : \partial H_0 \to T^2 \times \{0\}$ is the identity
map and
$f_1 : \partial H_1 \to T^2 \times \{1\}$ is
$$f_1 = \binom{0~~1}{1~~0}.$$
Let $\ell_i$ and $m_i$ be the longitude and meridian of $H_i$.
Let $\ell$ and $m$ be the longitude and meridian of $T^2 \times I$.

The specialized skein modules of the handlebodies,
$K_{-1}(H_0)$ and $K_{-1}(H_1)$, correspond to subvarieties of
the specialized skein module of the surface,
$K_{-1}(T^2)$.
\end{slide}

\begin{slide}
When we push a framed link from $T^2 \times I$ into
one of the handlebodies, there are relations induced by the fact that
$m_i$ is trivial and $\ell_i \simeq \ell_i m_i$ in handlebody $H_i$.

Recall from Example \ref{ex-torus} that $K_{-1}(T^2)$ is
generated by $x = -\tr(m)$, $y = -\tr(\ell)$, and $z = -\tr(\ell m)$.
Indeed,
$K_{-1}(T^2) = \bbc[x,y,z] / I$ where $I$ is the ideal
$I = (x^2+y^2+z^2+xyz-4)$.

The relations induced by $m_0 \simeq *$ and $\ell_0 \simeq \ell_0 m_0$
are $x = -\tr(m) = -\tr(m_0) = -2$ and
$y = -\tr(\ell) = -\tr(\ell_0) = -\tr(\ell_0 m_0) = -\tr(\ell m) = z.$
Let $J = (x+2, y-z)$, then $K_{-1}(H_0) = K_{-1}(T^2) / J$.

Similarly, the relations induced by $m_1 \simeq *$ and
$\ell_1 \simeq \ell_1 m_1$ are $y = -\tr(\ell) = -\tr(m_1) = -2$ and
$x = -\tr(m) = -\tr(\ell_1) = -\tr(\ell_1 m_1) = -\tr(\ell m) = z.$
Let $K = (y+2, x-z)$, then $K_{-1}(H_1) = K_{-1}(T^2) / K$.
\end{slide}

\begin{slide}
\begin{remark}
We are abusing
notation when we use $\tr(m)$, $\tr(\ell)$, etc.  We are suppressing
the representations $$\rho : \pi_1(H_i) \to \slc$$ and
$$\hat{\rho} : \pi_1(T^2) \to \slc.$$

The relations on the traces of the
matrices come from the curves themselves.  Thus it seems more instructive
to emphasize the curves over the matrices.
\end{remark}
\end{slide}

\begin{slide}
In order to show that there is no torsion, we need to
show that the maps
\[\Delta^r_1 :
\mathrm{Tor}^{K_{-1}(T^2)}_1 \Big( K_{-1}(H_1), K_{-1}(H_0) \Big)\]
\[
\to
\mathrm{Tor}^{K_{-1}(T^2)}_0 \Big( K_{-1}(H_1), K_{-1}(H_0) \Big)\]
are all zero maps.

Let $A^{\prime} = K_{-1}(T^2)$, $B^{\prime}_0 = K_{-1}(H_0)$, and
$B^{\prime}_1 = K_{-1}(H_1)$.  Then
\[\hspace{-1cm} \mathrm{Tor}^{K_{-1}(T^2)}_1 \Big( K_{-1}(H_1), K_{-1}(H_0) \Big) =
\mathrm{Tor}_1^{A^{\prime}} \Big( A^{\prime} / K, A^{\prime} / J \Big) \]
and we know from
homological algebra that
\[
\mathrm{Tor}_1^{A^{\prime}} \Big( A^{\prime} / K, A^{\prime} / J \Big)
= \frac{J \cap K}{JK}.
\]

Thus we begin by looking at $(J \cap K)/(JK)$ and then we will look at the
maps $\Delta^r_1$ in this context.
\end{slide}

\begin{slide}
\begin{lemma}
Let $J = (x+2, y-z)$ and $K = (y+2, x-z)$ be ideals of
$\bbc[x,y,z]$ and let $\gamma = y + 2 + x - z$.
Then, as a vector space over $\bbc$,
$$(J \cap K)/(JK)$$ is spanned by the set
$\{ \gamma, y \gamma, y^2 \gamma, y^3 \gamma, \dots \}$.
\label{lem-y-gamma}
\end{lemma}

\proof
Take $\alpha \in J \cap K$.  Then
\begin{eqnarray}
\alpha \in J & \Rightarrow & \alpha = p_1(x,y,z) (x+2) + p_2(x,y,z) (y-z) \nonumber \\
& \Rightarrow & \alpha(-2,y,y) = 0 \nonumber
\end{eqnarray}
and
\begin{eqnarray}
\alpha \in K & \Rightarrow & \alpha = q_1(x,y,z) (y+2) + q_2(x,y,z) (x-z). \nonumber
\end{eqnarray}
We know that
\begin{eqnarray}
JK & = & \Big( (x+2)(y+2), (y-z)(y+2), \nonumber \\
& & (x+2)(x-z), (y-z)(x-z) \Big). \nonumber
\end{eqnarray}
\end{slide}

\begin{slide}
We can use these to see that
\begin{eqnarray}
\alpha &  = & \Big( \tilde{q}_1(y) (y+2) + \tilde{q}_2(y) (x-z) \Big)\nonumber
\end{eqnarray}
as an element of $(J \cap K) / (JK)$.
Thus
\begin{eqnarray}
0 = \alpha(-2,y,y) & = & \tilde{q}_1(y)(y+2) + \tilde{q}_2(y)(-2-y) \nonumber \\
& = & (y+2)(\tilde{q}_1(y) - \tilde{q}_2(y)). \nonumber
\end{eqnarray}
This implies that $\tilde{q}_1(y)  = \tilde{q}_2(y)$ and so
$$\alpha = \tilde{q}_1(y) (y + 2 + x - z) = \tilde{q}_1(y) \gamma.$$
Thus, any element in $(J \cap K) / (JK)$ can be written as a linear
combination of elements from the set $\{ \gamma, y \gamma, y^2 \gamma, \dots \}$.
\qed
\end{slide}

\begin{slide}
\begin{theorem}
There is no torsion in $\overline{K(S^3)}$.
\label{thm-torsion-s3}
\end{theorem}

\proof
In $K_{-1}(T^2) = A^{\prime} = A / (1+t) A$, the
element $\gamma = y + 2 + x - z$ is equal to
$\tilde{\gamma} = y + [2] + x + t^3z$ where $[2] = t^2 + t^{-2}$.
In addition, we can push a skein from either handlebody into a neighborhood
of the torus.  Thus an element in
$\mathrm{Tor}_1^{K_{-1}(T^2)} \Big( K_{-1}(H_1), K_{-1}(H_0) \Big)$
can be written as a
linear combination of elements of the form $\phi \ot \alpha \ot \phi$, where
$\alpha$ is a linear combination of elements from the set
$\{ \tilde{\gamma}, y \tilde{\gamma}, y^2 \tilde{\gamma}, \dots \}$.

The tensor product and the $\Delta^r_1$ maps are both linear.  Thus
\begin{eqnarray}
\Delta^r_1(\phi \ot y^k \tilde{\gamma} \ot \phi) \nonumber \\
& & \hspace{-5cm} = \Delta^r_1(\phi \ot y^k (y + [2] + x + t^3 z) \ot \phi) \nonumber \\
& & \hspace{-5cm} = \Delta^r_1(\phi \ot y^{k+1} \ot \phi)
+ \Delta^r_1(\phi \ot y^k [2] \ot \phi) \nonumber \\
& & \hspace{-4cm} + \Delta^r_1(\phi \ot y^k x \ot \phi)
+ \Delta^r_1(\phi \ot y^k t^3 z \ot \phi). \nonumber
\end{eqnarray}
\end{slide}

\begin{slide}
There is a slight abuse of notation here.  We are using the
variables $x, y, z$ and their respective curves $m, \ell, \ell m$ interchangeably.
Recall that $x = -\tr(m)$, $y = -\tr(\ell)$, and $z = -\tr(\ell m)$.

Now we look at each of the four terms in the equation above.

\begin{eqnarray}
\Delta^r_1(\phi \ot y^{k+1} \ot \phi) & = &
x^{k+1} \ot \phi - \phi \ot y^{k+1} \nonumber \\
& = & (-[2])^{k+1} \phi \ot \phi - \phi \ot y^{k+1} \nonumber
\end{eqnarray}

\begin{eqnarray}
\Delta^r_1(\phi \ot y^k [2] \ot \phi) & = &
x^k [2] \ot \phi - \phi \ot y^k [2] \nonumber \\
& = & (-[2])^k[2] \phi \ot \phi - \phi \ot y^k [2] \nonumber
\end{eqnarray}

\begin{eqnarray}
\Delta^r_1(\phi \ot y^k x \ot \phi) & = &
y * x^k \ot \phi - \phi \ot y^k * x \nonumber \\
& = & (-[2])^k y \ot \phi - \phi \ot y^k (-[2]) \nonumber
\end{eqnarray}
\end{slide}

\begin{slide}
\begin{eqnarray}
\Delta^r_1(\phi \ot y^k t^3 z \ot \phi) & = &
t^3 z * x^k \ot \phi - \phi \ot t^3 y^k * z \nonumber \\
& & \hspace{-6cm} = (-[2])^k t^3 z \ot \phi - \phi \ot t^3 y^k * z \nonumber \\
& & \hspace{-6cm} = (-[2])^k t^3 (-t^{-3}) y \ot \phi -
\phi \ot t^3 (-t^{-3}) y^{k+1} \nonumber \\
& & \hspace{-6cm} = -(-[2])^k y \ot \phi - \phi \ot (-1) y^{k+1} \nonumber
\end{eqnarray}

The eight terms in the equations above
cancel in pairs.  Thus $\Delta^r_1(\phi \ot y^k \tilde{\gamma} \ot \phi) = 0$
for all $r$ and for all $k$.  Theorem \ref{thm-zero-maps} now applies to this Heegaard
splitting for $S^3$.  Thus there is no torsion in $\overline{K(S^3)}$.
\qed
\end{slide}

\begin{slide}
Take the genus one Heegaard splitting
$$H_1 \cup_{f_1} T^2 \times I \cup_{f_0} H_0$$ for $S^1 \times S^2$ where
$f_0 : \partial H_0 \to T^2 \times \{0\}$ is the identity
map and
$f_1 : \partial H_1 \to T^2 \times \{1\}$ is
$$f_1 = \binom{-1~~0}{~~0~~1}.$$

Consider the element $$\alpha \in
\mathrm{Tor}_1^{K_{-1}(T^2)} \Big( K_{-1}(H_1), K_{-1}(H_0) \Big)$$ given by
\begin{eqnarray}
\alpha & = & \phi \ot (p(t)*y - q(t)*z) \ot \phi \nonumber \\
& = & \phi \ot \Big( p(t)*\torusL - q(t)*\torusLM \Big) \ot \phi \nonumber
\end{eqnarray}
with $p(-1) = q(-1) = 1$.
\end{slide}

\begin{slide}
Now consider $\Delta^1_1(\alpha)$ which is given by
\begin{eqnarray}
\Delta^1_1(\alpha) & = & \Big( p(t)*\torusL - q(t)*\torusLMinv \Big) \ot
\phi \nonumber \\
& & - \phi \ot \Big( p(t)*\torusL - q(t)*\torusLM \Big) \nonumber \\
& = & (p(t)+t^3q(t))~\torusL \ot \phi \nonumber \\
& & - \phi \ot (p(t)+t^{-3}q(t))~\torusL. \nonumber
\end{eqnarray}

Since both $(p(t)+t^3q(t))$ and $(p(t)+t^{-3}q(t))$ are divisible by $(1+t)$, the
element $\alpha$ is a cycle and thus represents a nontrivial element in
$\mathrm{Tor}_1$.  However, $(p(t)+t^3q(t))$ and $(p(t)+t^{-3}q(t))$ are
not divisible by $(1+t)^2$, so $\Delta^1_1(\alpha) \neq 0$.
\end{slide}

\begin{slide}
Theorem \ref{thm-zero-maps} does not apply, but if we mimic the proof
of Theorem \ref{thm-zero-maps} for this example
we will see that at the $E^2$ level of the spectral sequence the term
$E^2_{-1,1}$ is no longer isomorphic to $E^1_{-1,1}$.

Instead it is the nontrivial
quotient of $E^1_{-1,1}$ by the image of $E^1_{0,1}$ under $\Delta^1_1$.  Compare
Figure 20.  Thus at the $E^{\infty}$ level, the map
\[
\frac{K(S^1 \times S^2)}{(1+t)K(S^1 \times S^2)} \to
\frac{(1+t)K(S^1 \times S^2)}{(1+t)^2K(S^1 \times S^2)}
\]
has nontrivial kernel.

Thus there exists an element $\beta \in K(S^1 \times S^2)$
that is not divisible by $(1+t)$ such that $(1+t) \beta$ is divisible by $(1+t)^2$.
Thus $(1+t) \beta = 0$ and torsion exists in $K(S^1 \times S^2)$.

The existence of torsion in $K(S^1 \times S^2)$ was shown by Hoste and Przytycki.
The above argument provides an alternate proof of this fact.
\end{slide}

\begin{slide}

Take the genus one Heegaard splitting
$$H_1 \cup_{f_1} T^2 \times I \cup_{f_0} H_0$$ for $\mathbb{RP}^3$ where
$f_0 : \partial H_0 \to T^2 \times \{0\}$ is the identity
map and
$f_1 : \partial H_1 \to T^2 \times \{1\}$ is
$$f_1 = \binom{1~~2}{1~~1}.$$

As in the case of $S^3$, $f_0$ is the identity, so we have
$J = (x+2, y-z)$ and $K_{-1}(H_0) = K_{-1}(T^2) / J$.
The inverse of the gluing map for $H_1$ is
\[f^{-1}_1 = \binom{-1~~~~~~2}{~~~1~~-1}.\]  Thus the inclusion of $T^2$ into $H_1$
sends $\ell^2 m$ to $m_1$, $\ell$ to $\ell_1^{-1} m_1$, and
$\ell m$ to $\ell_1$.
\end{slide}

\begin{slide}
The relation induced by $m_1 \simeq *$ uses $f_1^{-1}$.
In particular, $m_1 \simeq *$ induces
$\tr(\ell^2 m) = \tr(m_1) = 2$.  Using the trace identity for
$\slc$ we have
$\tr(\ell^2 m) = \tr(\ell)*\tr(\ell m) - \tr(m)$.
Thus $\tr(\ell^2 m) = 2$ becomes $yz+x = 2.$

Similarly a relation is induced
by $\ell_1 \simeq \ell_1 m_1 \simeq \ell_1^{-1} m_1$ using $f_1^{-1}$.
In particular,
\[y = -\tr(\ell) = -\tr(\ell_1^{-1} m_1) = -\tr(\ell_1) = -\tr(\ell m) = z.\]
Let $K = (yz+x-2, y-z)$, then $K_{-1}(H_1) = K_{-1}(T^2)$.
\end{slide}

\begin{slide}
\begin{lemma}
As a vector space over $\bbc$, $$\mathrm{Tor}_1^{K_{-1}(T^2)} \Big( K_{-1}(H_1),
K_{-1}(H_0) \Big)$$ is spanned by the set $\{ (y-z), y(y-z) \}$.
\end{lemma}

\proof
We know that $$\mathrm{Tor}_1^{K_{-1}(T^2)} \Big( K_{-1}(H_1),
K_{-1}(H_0) \Big)$$ is equal to $(J \cap K)/(JK)$ with $J = (x+2, y-z)$ and
$K = (yz+x-2, y-z)$.

Take $\alpha(x,y,z) \in (J \cap K)$.
We know that \[ \alpha(x,y,z) = p_1(x,y,z) (x+2) + p_2(x,y,z) (y-z) \] and
\[ \alpha(x,y,z) = q_1(x,y,z)(yz+x-2) + q_2(x,y,z) (y-z). \]
Use $(x+2)(yz+x-2) \in JK$ and $(y-z)(yz+x-2) \in JK$ to write $q_1(x,y,z)$
as a function in $y$.    Use $(x+2)(y-z)$ and $(y-z)(y-z)$ to write
$q_2(x,y,z)$ as a function in $y$.
\end{slide}

\begin{slide}
Thus
\[ \alpha(x,y,z) = \tilde{q}_1(y)(yz+x-2) + \tilde{q}_2(y)(y-z) \]
in the quotient $(J \cap K)/(JK)$.
Evaluating $\alpha(x,y,z)$ at $(-2,y,y)$ we have
\begin{eqnarray}
0 & = & \alpha(-2,y,y) \nonumber \\
& = & \tilde{q}_1(y)(y^2-4) + \tilde{q}_2(y)(y-y)\nonumber \\
& = & \tilde{q}_1(y)(y^2-4).\nonumber
\end{eqnarray}

Thus $\tilde{q}_1(y) = 0$ and $\alpha(x,y,z) = \tilde{q}_2(y)(y-z)$.

Hence
$(J \cap K)/(JK)$ is spanned by the set
$$\{ (y-z), y(y-z), y^2(y-z), y^3(y-z), \dots \}.$$
\end{slide}

\begin{slide}
Now consider the element $y^2(y-z) - 4(y-z)$ in $(J \cap K)/(JK)$ as follows.
\begin{eqnarray}
y^2(y-z) - 4(y-z) & & \nonumber \\
& & \hspace{-7cm} = y^2(y-z) - 2(y-z) - 2(y-z) + (x+2)(y-z)\nonumber \\
& & \hspace{-7cm} = y^2(y-z) - y(y-z)(y-z) - 2(y-z) + x(y-z)\nonumber \\
& & \hspace{-7cm} = y^2(y-z) - y^2(y-z) + yz(y-z) \nonumber \\
& & \hspace{-6cm} - 2(y-z) + x(y-z)\nonumber \\
& & \hspace{-7cm} = (yz + x - 2)(y-z).\nonumber
\end{eqnarray}

Since $(yz+x-2)(y-z) \in JK$, we have $$y^2(y-z) = 4(y-z)$$ in $(J \cap K)/(JK)$.
Thus $(J \cap K)/(JK)$ is spanned by the set $$\{ (y-z), y(y-z) \}.$$
\qed
\end{slide}

\begin{slide}
\begin{theorem}
There is no torsion in $\overline{K(\mathbb{RP}^3)}$.
\end{theorem}

\proof  As in the proof of Theorem \ref{thm-torsion-s3} any element
in $(J \cap K)/(JK)$ can be written as $\phi \ot \alpha \ot \phi$
where $\alpha$ is a linear combination of $(y-z)$ and $y(y-z)$.  To show
that $\Delta_1^r (\phi \ot \alpha \ot \phi) = 0$ it is enough to show that
$\Delta_1^r (\phi \ot (y-z) \ot \phi) = 0$ and $\Delta_1^r (\phi \ot y(y-z) \ot \phi) = 0$.

In $A / (1+t) A$ the element $y-z$ is equal to $y+t^3z$.
\begin{eqnarray}
\Delta_1^r ( \phi \ot (y + t^3 z) \ot \phi ) \nonumber \\
& & \hspace{-8cm} = f_1^{-1} ( \ell + t^3 \ell m) \ot \phi -
\phi \ot (\ell_0 + t^3 \ell_0 m_0) \nonumber \\
& & \hspace{-8cm} = (\ell_1^{-1} m_1 + t^3 \ell_1) \ot \phi -
\phi \ot (\ell_0 + t^3 \ell_0 m_0 ) \nonumber \\
& & \hspace{-8cm} = (-t^3 \ell_1 + t^3 \ell_1) \ot \phi -
\phi \ot (\ell_0 +(t^3)(-t^{-3}) \ell_0) \nonumber \\
& & \hspace{-8cm} = 0 \ot \phi - \phi \ot 0 \nonumber \\
& & \hspace{-8cm} = 0 \nonumber
\end{eqnarray}
\end{slide}

\begin{slide}
In $A / (1+t) A$, the element $y(y-z)$ is equal to
$$\beta = p(t)*y^2 - \frac{1}{2} yz - \frac{1}{2} zy$$
where $$p(t) = - \frac{1}{2} t^{-3} - \frac{1}{2} t^{-5}.$$
Apply $\Delta^r_1$ to the element $\phi \ot \beta \ot \phi$.

\begin{eqnarray}
\Delta^r_1 ( \phi \ot \beta \ot \phi) \nonumber \\
& & \hspace{-6cm} =
\Delta^r_1 \Big( \phi \ot \Big( p(t) y^2 - {\textstyle \frac{1}{2}} yz - {\textstyle \frac{1}{2}} zy \Big) \ot \phi \Big) \nonumber \\
& & \hspace{-6cm} =  f_1^{-1} \Big( p(t) \ell*\ell - {\textstyle \frac{1}{2}} \ell * \ell m - {\textstyle \frac{1}{2}} \ell m * \ell \Big) \ot \phi \nonumber \\
& & \hspace{-5cm} - \phi \ot \Big( p(t) \ell_0 * \ell_0 - {\textstyle \frac{1}{2}} \ell_0 * \ell_0 m_0 - {\textstyle \frac{1}{2}} \ell_0 m_0 * \ell_0 \Big) \nonumber \\
& & \hspace{-6cm} = \Big( p(t) \ell_1 m_1^{-1} * \ell m_1^{-1} - {\textstyle \frac{1}{2}} \ell_1 * \ell_1 m_1^{-1}
- {\textstyle \frac{1}{2}} \ell_1 m_1^{-1} * \ell_1 \Big) \ot \phi \nonumber \\
& & \hspace{-5cm} - \phi \ot \Big( p(t)*\ell_0 * \ell_0 - {\textstyle \frac{1}{2}} \ell_0 * \ell_0 m_0 - {\textstyle \frac{1}{2}} \ell_0 m_0 * \ell_0 \Big) \nonumber
\end{eqnarray}
\end{slide}

\begin{slide}
Let $\gamma = \ell_1 m_1^{-1} * \ell_1 m_1^{-1}$ as shown in Figure 22.
Removing the kinks in $\gamma$, we see that $\gamma = t^6 \delta$ where $\delta$
is the link shown in Figure 23.  Note also that $\ell_1 m_1^{-1} * \ell_1 = -t^3 \delta$
and $\ell_0 m_0 * \ell_0 = -t^{-3} \bar{\delta}$ where $\bar{\delta}$ is the mirror image of
$\delta$.

  \begin{center}
    \epsfxsize = 4cm
    \epsfysize = 2cm
    \epsfbox{figs/gamma.eps}

Figure 22:  The link $\gamma$ in a solid torus
  \end{center}

\vspace{2cm}

  \begin{center}
    \epsfxsize = 4cm
    \epsfysize = 2cm
    \epsfbox{figs/delta.eps}

Figure 23:  The link $\delta$ in a solid torus
  \end{center}
\end{slide}

\begin{slide}

Removing kinks and using $\gamma$ and $\delta$, the equation above becomes
\begin{eqnarray}
\Delta^r_1( \phi \ot \beta \ot \phi ) & = &
\Big( p(t) t^6 \delta + \LittleOneHalf t^3 \ell_1 * \ell_1 + \LittleOneHalf t^3 \delta \Big) \ot \phi \nonumber \\
& & \hspace{-6cm} - \phi \ot \Big( p(t) \ell_0 * \ell_0 + \LittleOneHalf t^{-3} \ell_0 * \ell_0 + \LittleOneHalf t^{-3} \bar{\delta} \Big) \nonumber
\end{eqnarray}
Now apply the skein relations to $\delta$ as shown in Figure 24.
In handlebody $H_1$ we have
$$\delta = t^2 \ell_1 * \ell_1 + (t^{-4} - 1)[2] \phi$$ and in handlebody $H_0$ we have
$$\bar{\delta} = t^{-2} \ell_0 * \ell_0 + (t^4 - 1)[2] \phi.$$
\end{slide}

\begin{slide}
  \begin{center}
    \epsfxsize = 12cm
    \epsfysize = 5cm
    \epsfbox{figs/skein-delta.eps}
    \put(-280, 123){$=~~~t$}
    \put(-185, 123){$+~t^{-1}$}
    \put(-280, 68){$=~~t^2$}
    \put(-185, 68){$~+$}
    \put(-90,68){$-~t^{-4}$}
    \put(-280,11){$=~~t^2$}
    \put(-185,11){$+~~(t^{-4} - 1) [2]$}

Figure 24:  The skein relations applied to the link $\delta$
  \end{center}
\end{slide}

\begin{slide}
Let
$$ \beta_1 = p(t) t^6 \delta + \LittleOneHalf t^3 \ell_1 * \ell_1 + \LittleOneHalf t^3 \delta$$
and
$$ \beta_0 = p(t) \ell_0 * \ell_0 + \LittleOneHalf t^{-3} \ell_0 * \ell_0 + \LittleOneHalf t^{-3} \delta.$$
Then $$\Delta( \phi \ot \beta \ot \phi ) = \beta_1 \ot \phi - \phi \ot \beta_0.$$
Consider
\begin{eqnarray}
\beta_1 & = & p(t) \Big( t^8 \ell_1 * \ell_1 + t^6(t^{-4} - 1)[2] \phi \Big) \nonumber \\
& & +
\LittleOneHalf t^3 \ell_1 * \ell_1 + \LittleOneHalf t^5 \ell_1 * \ell_1 +
\LittleOneHalf t^3(t^{-4} - 1)[2] \phi \nonumber \\
& = & (-\LittleOneHalf t^{-3} - \LittleOneHalf t^{-5})t^6(t^{-4} - 1)[2] \phi \nonumber \\
& & + \LittleOneHalf t^3(t^{-4} - 1)[2] \phi \nonumber \\
& = & - \LittleOneHalf t (t^{-4} - 1)[2] \phi \nonumber
\end{eqnarray}
and
\begin{eqnarray}
\beta_0 & = & p(t) \ell_0 * \ell_0 +
\LittleOneHalf t^{-3} \ell_0 * \ell_0 \nonumber \\
& & + \LittleOneHalf t^{-3} \Big( t^{-2} \ell_0 * \ell_0 +
(t^{4} - 1)[2] \phi \Big) \nonumber \\
& = & \LittleOneHalf t^{-3}(t^{4} - 1)[2] \phi. \nonumber
\end{eqnarray}
\end{slide}

\begin{slide}
Then
\begin{eqnarray}
\Delta^r_1( \phi \ot \beta \ot \phi ) & = & \beta_1 \ot \phi - \phi \ot \beta_0 \nonumber \\
& & \hspace{-6cm} = -\LittleOneHalf t (t^{-4} - 1)[2] \phi \ot \phi - \phi \ot \LittleOneHalf t^{-3}(t^{4} -1)[2] \phi
\nonumber \\
& & \hspace{-6cm} =  (- \LittleOneHalf t^{-3} + \LittleOneHalf t)[2] \phi \ot \phi - \phi \ot (\LittleOneHalf t
- \LittleOneHalf t^{-3}) [2] \phi \nonumber \\
& & \hspace{-6cm} = 0. \nonumber
\end{eqnarray}

Therefore the $\Delta^r_1$ maps are all zero maps and by
Theorem \ref{thm-zero-maps} there is no torsion in
$\overline{K(\mathbb{RP}^3)}$.
\qed

\end{slide}

\begin{slide}
\textbf{part five - future research}

The spectral sequence described above is a technique
for determining the existence of torsion in the skein module of a $3$-manifold $M$.
We have used this technique to identify whether or not torsion exists
in the skein modules of a few simple lens spaces.
In the examples we worked, we start with a Heegaard splitting for the $3$-manifold and
we analyze
$\mathrm{Tor}_1^{A^{\prime}} ( A^{\prime} / K, A^{\prime} / J )$ as the
quotient $(J \cap K) / (JK)$, where $A^{\prime} = K_{-1}(F)$ and $J$ and $K$
are the ideals that define $K_{-1}(H_0)$ and $K_{-1}(H_1)$, respectively.

If the splitting surface is genus two or higher, the process of
analyzing $J$ and $K$ as ideals gets more complicated because the 
number of variables needed to characterize $K_{-1}(F)$ is
$6 g - 3$ when $F$ is a genus $g$ surface.
However, we can analyze $\mathrm{Tor}_1^{A^{\prime}} ( A^{\prime} / K,
A^{\prime} / J )$ using algebraic varieties.  In particular, we make the following
conjecture.
\end{slide}

\begin{slide}
\begin{conjecture}
Let $M$ be a integral homology sphere such that $K(M)$ is
finitely generated.  Then $K(M)$ is torsion free.
\label{conjecture}
\end{conjecture}

A rough idea for the proof is as follows.
Given a genus $g$ Heegaard splitting $H_1 \cup F \times I \cup H_0$ for
$M$, we know that $\pi_1(H_i) = F_g$, the free group on $g$ generators,
and $\pi_1(F - \{ \mathrm{disc} \} ) = F_{2g}$.  Let
$R(H_i) = \mathrm{Rep}(\pi_1(H_i), \slc )$ and
$R(F) = \mathrm{Rep}(\pi_1(F - \{ \mathrm{disc} \} ), \slc )$.

Since $K(M)$ is finitely generated, $R(H_1) \cap R(H_0)$ is finite, thus
every intersection occurs at an isolated point.  We can localize at each
point $\rho$ and use results from Serre to see that
$$\mathrm{Tor}_1^{R(F)} ( R(H_1), R(H_0))_{(\rho)} = 0$$ whenever
$\rho$ is an irreducible representation.
\end{slide}

\begin{slide}
Thus we only need to analyze
the abelian representations.  Since $M$ is an integral homology
sphere, any abelian representation must factor through $H_1(M) = 0$,
thus the only abelian representation is the trivial representation.

However, $\mathrm{Tor}_1^{A^{\prime}} ( A^{\prime} / K , A^{\prime} / J )$ is
characterized by character varieties, not representation varieties.  Thus we must
find a way to extend these interactions among representation varieties to
character varieties.
If we can extend to the character varieties, we should be able to show that the
$\Delta^r_1$ maps of Theorem \ref{thm-zero-maps} are all zero in this case.
Then we will have shown that $K(M)$ is torsion free.
\end{slide}

\end{document}

