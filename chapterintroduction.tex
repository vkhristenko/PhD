\chapter{Introduction} \label{chapter:introduction}
Before moving on to the actual research topics, here, in the first chapter we briefly outline the structure of this thesis and try to provide a very general overview of the topics to be discussed further in detail.

Modern Experimental High Energy Physics (HEP) is a \textbf{science at scale} from several points of view: the amount of money it requires to have invested, the physical dimensions of instruments that are built and number of people that collaborate on the same project. It is a unique field that brings in the skills across various domains and provides a unique opportunity to get involved in several independent projects that, in the end, come together as a single scientific result. This thesis is no different - the topics discussed are completely independent and fall into separate categories, however they fill the phase space of the research program for the Compact Muon Solenoid Experiment.

We get started by first presenting the results of the search for the Standard Model Higgs Boson decaying via two oppposite sign muons. The analysis performed is an extension of previous work that has been performed earlier with CMS Experiment during Run I campaign, however we benefit from increased center of mass energy and higher instantenous luminosity.

In the second part we introduce CMS Hadron Forward (HF) Calorimeter and discuss the calibration procedure using a radioactive source. During the Long Shutdown 1 (LS1) of the Large Hadron Collider (LHC), HF has undergone substantial electronics upgrade which required the energy scale of the calorimeter to be reestablished. We will dive into the details of experimental setup and analysis workflow. The success of the presented calibration technique and results has been demonstrated by the success of the CMS Run II data-taking and analysis campaigns.

Final chapter focuses on how on building simulations of calorimeters, evaluating basic performance characteristics and scaling them up to the production dimensions of your future experiment. We will focus on two particular examples: High Gradnularity Calorimeter (HGC) and Shashlik plus Hadron Endcap System. Both systems were potential candidates for the future upgrades of the CMS Endcap and constitute ideal benchmarks for our simulations.

%We get started by asking ourselves a question of how one would go about building a new experiment, a new system or something more concrete - a calorimeter. A possible answer would be to just build it first, however as we noted HEP is an expensive science and it would not be prudent to build something without knowing the performance characteristics of the system of interest. Therefore, second chapter focuses on how to build simulations of calorimeters, evaluate basic performance characteristics and scale them up to the production dimensions of your future experiment. We will focus on two particular examples: High Gradnularity Calorimeter (HGC) and Shashlik plus Hadron Endcap System. Both systems were potential candidates for the future upgrades of the CMS Endcap and constitute ideal benchmarks for our simulations.

%Moving forward, we consider a situation when one has already built a calorimeter. In chapter three, we will introduce Hadron Forward Calorimeter of CMS Experiment (we will defer the introduction to the CMS itself until the final chapter) and show how to calibrate our device and establish the energy scale to be used during the Reconstruction Procedure. We will describe the procedure via which the calibration is performed and identify potential problems for the future upgrades.

%Finally, given that we have simulated our detector system, calibrated it and spent several months (in case of CMS years) running it, we are left with the final step on our journey towards the physics result - we ought to perform the analysis of the data that we have collected. In the last chapter we present results of the search for Standard Model Higgs Boson decaying via opposite sign muons.