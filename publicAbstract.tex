Higgs Boson is the last missing piece of the most successful theoretical model of the elementary particle physics - Standard Model. The construction of the Large Hadron Collider and the experiments hosted by it was optimized with the search for the Higgs Boson in mind. The first part of this work focuses on trying to find the Higgs Boson in a dimuon decay channel with the Compact Muon Solenoid (CMS) detector.

% A search for the Standard Model Higgs Boson in a dimuon final state in proton-proton collisions with the Compact Muon Solenoid Experiment is performed. Building on top of the success of previous CMS analyses (CMS Run I campaign), results are presented using 35.9 fb$^{-1}$ of data collected over the course of 2016 (CMS Run II campaign) at a center-of-mass energy of $\sqrt{s} = 13$ TeV.

Calorimeter is a device for measuring energy. In High Energy Physics, calorimeters are responsible for determining the energy of the particles passing through. Like any other measuring system, it needs to be calibrated to associate properly its response to the actual energy units. Second part discusses the calibration process of the Hadron Forward calorimeter of the CMS experiment.
% During the Long Shutdown 1 of the Large Hadron Collider, CMS Experiment has undergone substantial hardware changes. Second topic discusses the process of Calibration of the CMS Hadron Forward Calorimeter in preparation for collisions after LS1.

Final chapter looks at simulations of calorimeters proposed for the future upgrades. Two different systems are built and evaluated. The focus is on obtaining basic performance characteristics.

% Final chapter discusses the process of building simulations of calorimeter systems. Walking through all the steps from geometry specification to readout definition the results for two standalone calorimeters are presented that have been proposed as potential replacements for respective CMS components.