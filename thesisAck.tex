First of all, I would like to express my deep gratitude to my advisor, professor Yasar Onel, for the possibility to be his student. In particular, I want to note the amount of experience, knowledge and connections I have made by staying at CERN for the past three years. It has been an incredible experience and would not be possible without his continuous support and advice.

I would like to thank Ugur Akgun for introducing me into the world of High Energy Physics. For me it has all started as a summer undergraduate student working on various detector simulations. My time at the University of Iowa and at CERN would not be the same without James Wetzel and David Southwick. From arguing about current politics and going out in Geneva to the actual physics discussions - it has been the one experience I will never forget.

I would like to thank Alexi Mestverishvili, who has made the best soup I have ever eaten in my life. For the friendship and kindness that he has shown to a poor student coming to Geneva to work on Compact Muon Solenoid Experiment.

I would like to thank one of the best detector physicists I have ever met - Burak Bilki, who did not just teach me how to perform simulations of calorimeters and set up a trigger system for the Fermilab Test Beam Experiment, but who has also been a good friend!

I would like to acknowledge professor Jane Nachtman, Ianos Schmidt, Paul Debbins and the rest of the Iowa group for the numerous discussions and help I have received during my research.

I would like thank professor Darin Acosta for joining my committee and recognize my colleagues from the Higgs group: Darin Acosta, Sergei Gleyzer, Adrian Perieanu, Andrew Brinkerhoff, Andrea Marini, Andrew Carnes, Pierliugi Bortignon. I would like to acknowledge them not just for the possibility to join this wonderful analysis team, but, what is more important to me, it is through the interaction with these people that I have been able to finally understand, appreciate and enjoy the process of doing a physics analysis.

I would like to thank professors Craig Pryor and Markus Wohlgenannt, who agreed to join my thesis committee, provide continuous support for my research program and for overall cooperation with me on various logistics matters.

I would like to share the pleasure I have had to be part of the Physics Department at the University of Iowa for, first of, the opportunity to just be a member of this incredible group of people; and, secondly, for the support of the whole staff over these years. That has made a fundamental difference in my experience as a student.

I would like to acknowledge members of the HCAL personnel for the possibility of working side-by-side on such an exciting project, learning directly from the most experienced people in the field, and for the amazing barbecue parties organized over the years.

I would like to thank my undergraduate physics professors Steve Feller and Mario Affatigato for running Coe College Physics program as a family business!

Going across the globe, I want to recognize people who taught me not just how to program or to solve differential equations, but how to actually think rationally and how to approach solving various problems, most importantly in life! I want to thank my high school computer science and mathematics teachers Konstantin Krivonogov and Vladimir Tupin.

I would like to thank my friends in Iowa: Eric Rodgers; Kris and Phil Diehl; Steve Eden; Marina Stankovich and Gary Dixon; the Postnikov family - for the friendship that I will try to cherish and carry over with me through my life.

Finally, I would like to mention my friends and family from Russia - my parents, uncles and aunts, my grandparents, my younger brother, my cousins, my best friends, my girlfriend Kate - who taught me how to enjoy simple things. These are the people who have always been behind my back and supported me to keep moving forward.